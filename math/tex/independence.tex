\M
Suppose we have two events $X$ and $Y$. If $Y$ does not depend on $X$,
we expect
\begin{equation}\label{eq:firstAttemptIndependence}
\Pr(Y|X)=\Pr(Y).
\end{equation}
Lets try to consider a slightly more general situation. If we multiply
both sides of Eq \eqref{eq:firstAttemptIndependence} by $\Pr(X)$, we get
\begin{equation}
\begin{split}
\Pr(Y|X)\Pr(X)&=\Pr(X\cap Y)\\
&=\Pr(Y)\Pr(X)
\end{split}
\end{equation}
which gives us the desired condition for ``independence.''

\N{Definition}
Let $X$, $Y$ be events. We call them \define{Independent Events} if and
only if 
\begin{equation}
\Pr(X\cap Y)=\Pr(X)\Pr(Y).
\end{equation}
More generally, for a family of events $X_{i}$, they are independent iff
\begin{equation}
\Pr\left(\bigcap_{j}X_{j}\right)=\prod_{j}\Pr(X_{j})
\end{equation}

Why is this a good definition? If $X$ or $Y$ has probability zero, we
avoid the risk of dividing by zero. This could not have been avoided
using Eq \eqref{eq:firstAttemptIndependence}. But notice our condition
for independence implies Eq \eqref{eq:firstAttemptIndependence}!

\N*{Caution:} Do not make the rookie mistake thinking, for a family of
events $X_{j}$, independence holds iff for each $i\neq j$ we have
$\Pr(X_{i}\cap X_j)=\Pr(X_i)\Pr(X_j)$. This is \emph{pairwise
independence}, and not necessarily the same as implying the family
consists of independent events.

\N{Example (Pairwise Independence Problems)}
Suppose we have 
\begin{equation}
\sampleSpace = \{ abc,acb,cab,cba,cab,bca,bac,aaa,bbb,ccc \}
\end{equation}
and they are all equal probable outcomes. Let $A_{k}$ be the event the
$k$th letter is $a$. 

We claim $\{A_1,A_2,A_3\}$ is a family of pairwise independent
events. Observe each $A_k$ has three events. For example,
$A_1=\{abc,acb,aaa\}$. Then we see
\begin{equation}
\Pr(A_{i}\cap A_{j})=\Pr(aaa)=\frac{1}{9}
\end{equation}
and
\begin{equation}
\Pr(A_i)\Pr(A_j)=\frac{1}{3}\cdot\frac{1}{3}=\frac{1}{9}.
\end{equation}
Thus we see
\begin{equation}
\Pr(A_i\cap A_j)=\Pr(A_i)\Pr(A_j)
\end{equation}
for $i\neq j$. This is the definition of pair-wise independent.

However, observe
\begin{equation}
\Pr(A_1\cap A_2\cap A_3)=\Pr(aaa)=\frac{1}{9}
\end{equation}
whereas
\begin{equation}
\Pr(A_1)\Pr(A_2)\Pr(A_3)=\frac{1}{27}.
\end{equation}
So we have
\begin{equation}
\Pr(A_1\cap A_2\cap A_3)\neq\Pr(A_1)\Pr(A_2)\Pr(A_3).
\end{equation}
That is to say, our family of pairwise-independent events is not a
family of independent events!
