\M
If we have a discrete random variable $X\colon\sampleSpace\to\RR$, we
can ask what's its \emph{expected value}? What does this mean? We mean,
if $X=x_{j}$ for $j\in\NN$, we want to consider the expression
\begin{equation}
\expected[X]=\sum_{j\in\NN}x_{j}\Pr(X=x_{j}).
\end{equation}
The intuition is that
\begin{equation}
\expected[X] = \frac{1}{N(\sampleSpace)}\sum_{j\in\NN}x_{j}N(x_{j})
\end{equation}
which is precisely what we have.

\N{Example}
Recall Example \ref{ex:studentGuessingOnExam} when a student guesses on
a true-false exam. What's the expected value of a student guessing on a
10 question true-false exam?

\N*{Solution:}
We see that
\begin{equation}
\expected[X] = \sum^{10}_{n=0}n\cdot\binom{10}{n}2^{-10}
\end{equation}
We recall
\begin{equation}
(1+x)^{n}=\sum^{n}_{k=0}\binom{n}{k}x^{k}
\end{equation}
thus taking its derivative gives us
\begin{equation}
n(1+x)^{n-1}=\sum^{n}_{k=0}k\binom{n}{k}x^{k-1}.
\end{equation}
We set $x=1$ and obtain
\begin{equation}
n2^{n-1}=\sum^{n}_{k=0}k\binom{n}{k}.
\end{equation}
We thus deduce
\begin{equation}
\expected[X]=2^{-10}\cdot 10\cdot2^{9}=5.
\end{equation}
An anticlimactic solution: guessing should give a score of 50\%.

\begin{thm}
The expectation operator satisfies the following properties:
\begin{enumerate}
\item If $X\geq0$, then $\expected[X]\geq0$
\item For any $a,b\in\RR$ we have
$\expected[aX+bY]=a\expected[X]+b\expected[Y]$
\item The random variable ${\bf 1}$ which takes the constant value $1$
satisfies $\expected[{\bf 1}]=1$.
\end{enumerate}
\end{thm}
\begin{proof}
(1) We see that if $X$ takes values $x_j\geq0$, then $\Pr(X=x_j)\geq0$,
and the product of two positive real numbers is positive. The sum of
positive real numbers is itself a positive real number.

(2) Linearity follows immediately:
\begin{subequations}
\begin{align}
\expected[aX+bY] &=\sum_{\omega}a\omega\Pr(X=\omega)+b\omega\Pr(Y=\omega)\\
&= \sum_{x}ax\Pr(X=x) + \sum_{y}by\Pr(Y=y)\\
&=a\sum_x x\Pr(X=x)+b\sum_y y\Pr(Y=y) = a\expected[X]+b\expected[Y].
\end{align}
\end{subequations}

(3) Obvious, since it becomes a sum of probabilities which must be unity.
\end{proof}

\subsection{Joint Distributions}
\N{Definition}
Let $X$, $Y$ be discrete random variables. Their \define{Joint
Distribution} $F\colon\RR^{2}\to[0,1]$ is given by
\begin{equation}
F(x,y)=\Pr(X\leq x, Y\leq y)
\end{equation}
and their \define{Joint Mass Function} is
\begin{equation}
f(x,y) = \Pr(X=x,Y=y).
\end{equation}
We sometimes use the notation $f_{X,Y}(x,y)$ to make it clear what the
random variables are.
\N{Definition}
We see that $X$ and $Y$ are \define{Independent} if and only if
\begin{subequations}
\begin{align}
f_{X,Y}(x,y)
&= \Pr(X=x,Y=y)\\
&=\Pr(X=x)\Pr(Y=y)\\
&=f_{X}(x)f_{Y}(y)
\end{align}
\end{subequations}
Note we induce this notion using the usual notion of independence.

\N{Definition}
We have the \define{Covariance} of $X$ and $Y$ be
\begin{equation}
\cov(X,Y)=\expected[XY]-\expected[X]\expected[Y]
\end{equation}
and the \define{Correlation Coefficient}
\begin{equation}
\rho(X,Y)=\frac{\cov(X,Y)}{\sqrt{\Var(X)\Var(Y)}}
\end{equation}
Recall we defined the variance $\Var(X)$ in \Mref{defn:variance}.

\begin{lemma}[Independence Condition]
Let $X$, $Y$ be random variables. They are independent if and only if 
\begin{equation}
\cov(X,Y)=0.
\end{equation}
\end{lemma}
\begin{proof}
We see that
\begin{equation}
\begin{split}
\expected[XY]-\expected[X]\expected[Y]
&=\sum_{x,y}xy\Pr(X=x,Y=y)-x\Pr(X=x)y\Pr(Y=y)\\
&=\sum_{x,y}xy\bigl(\Pr(X=x,Y=y)-\Pr(X=x)\Pr(Y=y)\bigr)
\end{split}
\end{equation}
but we see independence for joint distributions precisely occurs when
\begin{equation}
\bigl(\Pr(X=x,Y=y)-\Pr(X=x)\Pr(Y=y)\bigr)=0
\end{equation}
for any $x$ and $y$.
\end{proof}

\begin{thm}[Cauchy-Schwarz Inequality]
Let $X$, $Y$ be random variables. Then
\begin{equation}
\bigl(\expected[XY]\bigr)^{2}\leq\expected[X^{2}]\expected[Y^{2}]
\end{equation}
with equality if and only if $\Pr(aX=bY)=1$ for some $a,b\in\RR$ (at
least one of which is nonzero).
\end{thm}
\begin{proof}
We introduce a new random variable
\begin{equation}
Z = aX + bY
\end{equation}
where $a,b\in\RR$. Suppose $a\geq0$. Then
\begin{equation}
0\leq\expected[Z^{2}]=a^{2}\expected[X^{2}]+b^{2}\expected[Y^{2}]-2ab\expected{XY}.
\end{equation}
We consider the right hand side as a quadratic function in $a$. When
does it have a real root? When
\begin{subequations}\label{eq:Cauchy-Schwarz:prob:proof}
\begin{equation}
B^{2}-4AC\leq0
\end{equation}
or for us
\begin{equation}
4b^{2}\expected[XY]^{2}-4\expected[X^{2}]\cdot
b^{2}\expected[Y^{2}]\leq0
\end{equation}
\end{subequations}
For nonzero $b$, we have the desired result immediately.

Observe one real root of Eq \eqref{eq:Cauchy-Schwarz:prob:proof}
implies
\begin{subequations}
\begin{equation}
\expected[Z^{2}]=0
\end{equation}
which implies
\begin{equation}
Z=0\quad\mbox{with probability }1.
\end{equation}
Thus
\begin{equation}
aX-bY=0
\end{equation}
\end{subequations}
with probability 1.
\end{proof}

\subsection{Conditional Distributions and Expectations}

Let $X$ and $Y$ be two discrete random variables on
$(\sampleSpace, \mathcal{F}, \Pr)$.

\begin{defn}
The \define{Conditional Distribution Function} of $Y$ given $X=x$,
written $F_{Y|X}(-|x)$, is defined by
\begin{equation}
F_{Y|X}(y|x) = \Pr(y\leq Y|X=x)
\end{equation}
for any $x$ such that $\Pr(X=x)>0$.

The \define{Conditional Probability Function} (or ``\emph{Conditional
Mass Function\/}'') of $Y$ given $X=x$, written $f_{Y|X}(-|x)$, is
defined as
\begin{equation}
f_{Y|X}(y|x)=\Pr(Y=y|X=x)
\end{equation}
\end{defn}
\begin{rmk}
Note this definition implies
\begin{equation}
f_{Y|X}=\frac{f_{Y,X}}{f_{X}}
\end{equation}
and that $X$ and $Y$ are independent if and only if
\begin{equation}
f_{Y|X}=f_{Y}.
\end{equation}
This justifies the choice of notation.
\end{rmk}
\begin{rmk}
Given $X=x$, we may ``define''
\begin{equation}\label{eq:defn:rmk:condExpect}
\begin{split}
\expected[Y|X=x] &\eqdef \sum_{y}yf_{Y|X}(y|x)\\
&=\mbox{conditional expectation of $Y$ given $(X=x)$}
\end{split}
\end{equation}
We may think of this as a function of $x$:
\begin{equation}
\psi(x) = \expected[Y|X=x].
\end{equation}
Thus the expected value may be thought of as a function of $X$:
\begin{equation}
\psi(X) = \expected[Y|X].
\end{equation}
This is a mildly sloppy abuse of notation.
\end{rmk}
\begin{defn}
Let $\psi(x)=\expected[Y|X=x]$. Then the \define{Conditional
Expectation} of $Y$ given $X$ written $\expected[Y|X]$ is precisely
$\psi(X)$. Note that $\expected[Y|X]$ is a random variable. 
\end{defn}

\begin{thm}
Given random variables $X$, $Y$ as specified, then
\begin{equation}
\expected\bigl[\expected[Y|X]\bigr] = \expected[Y].
\end{equation}
\end{thm}
\begin{proof}
This is a classic ``follow-your-nose and unravel the definitions'' type proof.
\begin{subequations}
\begin{align}
\expected\bigl[\expected[Y|X]\bigr]
&=\sum_{x}f_{X}(x)\left(\sum_{y}yf_{Y|X}(y|x)\right)\\
&=\sum_{y}y\sum_{x}f_{X,Y}(x,y)\\
&=\sum_{y}yf_{Y}(y)=\expected[Y].
\end{align}
\end{subequations}
We just began with Eq \eqref{eq:defn:rmk:condExpect} and ``followed our nose''!
\end{proof}

Note more generally, we have
\begin{equation}
\expected\bigl[\expected[Y|X]g(X)\bigr]=\expected[Yg(X)].
\end{equation}
The proof is simple:
\begin{subequations}
\begin{align}
\expected\bigl[\psi(x)g(x)\bigr]
&=\sum_{x}\psi(x)g(x)\Pr(X=x)\\
&=\sum_{y}y\sum_{x}\Pr(Y=y|X=x)g(x)\Pr(X=x)\\
&=\sum_{x,y} yg(x)\Pr(X=x,Y=y)=\expected[Yg(X)].
\end{align}
\end{subequations}
% pg 15 of notes 7
\begin{thm}
Let $a,b\in\RR$, and $X$, $Y$, $Z$ be random variables.
\begin{enumerate}
\item $\expected[aY+bZ|X]=a\expected[Y|X]+b\expected[Z|X]$
\item $\expected[Y|X]\geq0$ if $Y\geq0$
\item $\expected[1|X]=1$
\item If $X$ and $Y$ are independent, then $\expected[Y|X]=\expected[Y]$
\item $\expected[Yg(X)|X]=g(X)\expected[Y|X]$ where $g\colon\RR\to\RR$
\item $\expected\bigl[\expected[Y|X,Z]|X\bigr]=\expected[Y|X]$ and $\expected\bigl[\expected[Y|X]|X,Z\bigr]=\expected[Y|X]$.
\end{enumerate}
\end{thm}
