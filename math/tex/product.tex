\N{Proposition}
Let $\mathcal{F}$, $\mathcal{G}$ be $\sigma$-algebras over
$\sampleSpace$. Then $\mathcal{F}\cap\mathcal{G}$ is also a
$\sigma$-algebra over $\sampleSpace$.

More generally, if $\mathcal{F}_{i}$ is a family of $\sigma$-algebras
over $\sampleSpace$, then
\begin{equation}
\bigcap_{i}\mathcal{F}_{i}=\widetilde{\mathcal{F}}
\end{equation}
is a $\sigma$-algebra over $\sampleSpace$.

\N{Problem:}\label{chunk:product:problem}
We claim that, if $\mathcal{F}$ and $\mathcal{G}$ are $\sigma$-algebras
over $\sampleSpace$, then $\mathcal{F}\cup\mathcal{G}$ is \emph{not} a
$\sigma$-algebra over $\sampleSpace$. No! 

But we can \emph{uniquely extend} $\mathcal{F}\cup\mathcal{G}$ to a
``smallest'' $\sigma$-algebra containing both $\mathcal{F}$ and
$\mathcal{G}$ as subalgebras. What to do? We simply consider the
collection
\begin{equation}
\{\mathcal{H}_{i} : \mathcal{F}\subset\mathcal{H}_{i},\quad\mbox{and}\quad
\mathcal{G}\subset\mathcal{H}_{i}\}
\end{equation}
then we construct
\begin{equation}
\bigcap_{i}\mathcal{H}_{i}=\mathcal{H}.
\end{equation}
This is the smallest such $\sigma$-algebra containing both $\mathcal{F}$
and $\mathcal{G}$.

\M
Recall we describe an experiment using a probability space. But what if
we want to have a ``composite'' experiment? Say, flip a coin \emph{and}
draw a card from a deck. How can we describe this experiment? Let us try
to consider it!

We want to combine $(\sampleSpace_1,\mathcal{F}_1,\Pr_1)$ and
$(\sampleSpace_2,\mathcal{F}_2,\Pr_2)$. What to do?

First lets construct the sample space. We expect, correctly, that
\begin{equation}
\sampleSpace=\sampleSpace_1\times\sampleSpace_2
\end{equation}
is our sample space. 

Next the $\sigma$-algebra. This is more subtle, and requires some
justification (given in our discussion of ``completeness''). The set
$\mathcal{F}_1\times\mathcal{F}_2$ \emph{is not} a $\sigma$-algebra. But
we can construct the smallest $\sigma$-algebra containing it! We use
this ``smallest'' $\sigma$-algebra. We use the same process outlined in \S\ref{chunk:product:problem}.

The probability measure is simply $\Pr_{12}(A_1\times
A_2)=\Pr_1(A_1)\Pr_2(A_2)$, where $A_1\in\mathcal{F}_1$ and
$A_2\in\mathcal{F}_2$. 
