\section{Dynamics}

\M
So far we have been setting up the language to describe motion
(``kinematics''). We will now start exploring the \emph{causes} of
motion (``dynamics'').

\subsection{Force and Interaction}

\begin{definition}
A \define{Force}\index{Force|textbf} is a push or a pull (or
\emph{any interaction which changes the motion of a body}). It describes the
interaction between a body and its environment, or between two bodies. A
force is a vectorial quantity involving its strength and direction.
\end{definition}

\begin{definition}
A \define{Contact Force}\index{Force!Contact} is a force that requires
direct contact between two bodies, like a push, a pull, or friction.
\end{definition}

\begin{ex}[Tension]
When a string, a cord, a rope is attached to an object, there is a force
exerted on the object when pulling the rope \emph{and} the rope is taut.
This is in the direction of the pull. Such a force is called
\define{Tension}\index{Tension|textbf}.
\end{ex}

\begin{ex}[Normal Force]
What prevents an object from falling through the ground? There is a
force the ground exerts on objects, called the
\define{Normal Force}\index{Normal Force|textbf}. More generally, if a
body is in contact with some surface, that surface exerts a normal force
on the body.
\end{ex}

\begin{ex}[Friction]
Friction occurs when dragging an object against the ground (or some
flooring). It is defined as $\vec{F}=\mu\vec{N}$ where $\mu$ is the
coefficient of friction, $\vec{N}$ is the normal force acting on the
object.
\end{ex}

\begin{ex}[Drag]
When an object moves through a fluid, the fluid ``pushes back'' against
the object. This exerts a force known as \define{Drag}. It is
proportional to the square of the magnitude of the velocity of the
moving body through the fluid, but it points in the opposite direction
as the velocity
\begin{equation}
\vec{F}_{\text{drag}}\looksLike v^{2}(-\widehat{\vec{v}})
\end{equation}
modulo some proportionality constant.
\end{ex}

\begin{ex}[Spring]
A spring connects two objects, and provides an oscillatory force between
the objects. Since springs require direct contact with bodies, it is
considered a contact force. Note: not all simple harmonic oscillators
stem from contact forces. A pendulum uses gravitational force for its
oscillatory movement, which is a long range force.
\end{ex}

\begin{definition}
A \define{Long Range Force}\index{Force!Long-Range} acts on a body or
bodies when separated by empty space. Gravity and electromagnetism are
examples of long range forces.
\end{definition}

\begin{ex}[{Fictitious Forces}]
In rotating or accelerating reference frames~\Mref{defn:newtonian-dynamics:inertial-reference-frame},
it appears as though there are ``extra forces'' at play. Such forces
(like the centrifugal force, Euler force, and the Coriolis effect) are
``long range forces'', if only because there is no acting body in
immediate contact with the recipient of the force.
\end{ex}

\N{Superposition}
When two forces $\vec{F}_{1}$ and $\vec{F}_{2}$ act on the same body,
then the effect is the same as a single force
$\vec{F}=\vec{F}_{1}+\vec{F}_{2}$ acting on that body. This is the
\define{Superposition of Forces}\index{Force!Superposition Principle}.

\begin{remark}[Empirically Lacking]
Michael Spivak~\cite{spivak} pointed out this ``superposition of forces''
does not appear to be experimentally verified, just assumed to work. How
would one experimentally test the superposition principle?
\end{remark}
\begin{remark}[Components of Force]
We can use the superposition of forces to break up a force into its
components.
\end{remark}

\begin{definition}
The \define{Net Force}\index{Force!Net}\index{Net Force} acting on a body is the sum of all the individual
acting forces acting on the body.
\end{definition}

\N{Detecting Force}
One method of detecting contact forces uses a ``Spring Balance''%
\index{Balance!Spring}\index{Spring Balance}\index{Force!Detecting}
which amounts to a spring hanging vertically with one end fixed, and a
``pointer'' attached to the other end. As the spring contracts and
expands, force must be expended, which is measured by the pointer's
vertical position. The author has never seen this done in the ``real world''.
Perhaps experimentalists don't like people watching\dots

\subsection{Newton's First Law}

\N{Law}\label{law:newton:first}\index{Newton's Laws!First}
A body acted on by no net force moves with constant velocity
(which may be zero) and zero acceleration.

\begin{definition}
The tendency for a body to resist motion is its \define{Inertia}\index{Inertia|textbf}.
\end{definition}

\M
Newton's first law speaks of zero \emph{net} force. A book on a table
feels two forces: the gravitational pull of the Earth, and the table's
push against the book. The upward supporting force of the table is the
\define{Normal Force}.

\N[{Neher and Leighton~\cite{neher}}]{Experiment}
Suppose we have an air hockey table. That is, a plastic puck supported
by gas and not sliding on the table directly. If Newton's first law
holds, then the puck's final velocity would be equal to its initial
velocity, and would travel a fixed distance in a known amount of time.

There is some subtlety here, because the gas introduces new physics
(stuff we have yet to introduce). Yet when adequately taken into
account, Newton's first law appears to hold.

\begin{definition}
A body is in \define{Equilibrium}\index{Equilibrium!for Body} if and only if the net force acting on
it vanishes
\begin{equation}
  \sum\vec{F} = \vec{0}.
\end{equation}
For this to be true, each component must vanish $\sum F_{x}=0$ and $\sum
F_{y}=0$.
\end{definition}

\begin{definition}\label{defn:newtonian-dynamics:inertial-reference-frame}
A \define{Inertial Reference Frame} consists of a reference frame such
that Newton's first law holds. A \define{Non-Inertial Reference Frame}
is one where Newton's first law fails.
\end{definition}

\begin{remark}[Non-Inertial Examples]
Newton's first law will not hold if the observer's reference frame is
accelerating, or if the observer is rotating about an axis. Arguably, an
observer on Earth cannot be inertial, since the Earth rotates constantly.
\end{remark}

\begin{theorem}
Let $A$ be an inertial reference frame, and $B$ be a reference frame
moving with constant velocity relative to $A$. Then $B$ is also an
inertial reference frame.
\end{theorem}
\begin{proof}[Sketch of Proof.]
If Newton's first law holds, then a free body moves with constant velocity
relative to $A$'s frame $\vec{v}_{A}$. But if $B$ moves with velocity
$\vec{v}_{B|A}$, then the free body moves with velocity
$\vec{v}_{B}=\vec{v}_{A}+\vec{v}_{B|A}$ which is the sum of two constant
vectors. This implies Newton's first law \Mref{law:newton:first} holds.
\end{proof}

\subsection{Newton's Second Law}

\M
Newton's first law \Mref{law:newton:first} describes what happens when
the net force on an object is zero (i.e., the object is in
equilibrium). But what happens when the net force is nonzero?

\begin{definition}
A body's \define{Inertial Mass}\index{Mass!Inertial|textbf} (or simply ``\emph{Mass\/}'') measures a
body's resistance to motion (``inertness''). We denote this quantity by $m$.
\end{definition}

\begin{remark}[SI units of mass, force]
We recall~\Mref{M:intro:standards-and-units} the SI unit of mass (the
kilogram) is measured by a certain platinum--iridium alloy rod kept in
Paris. We measure force using ``Newtons'', defined as the net force to
move 1 kilogram with an acceleration of 1 meter per second squared:
\begin{equation}
\SI{1}{\newton}=\SI{1}{\kilo\gram\meter\per\second\squared}.
\end{equation}
\end{remark}

\N{Law}\label{M:newton:second}\index{Newton's Laws!Second}
If a net force acts on a body, then the body accelerates. The direction
of acceleration is the same as the direction of the net force. The net
force is the body's inertial mass times acceleration
\begin{equation}
  \vec{F} = m\vec{a}
\end{equation}
\begin{remark}
  This is specifically in an inertial reference frame.
\end{remark}
\begin{remark}
  We can discuss the ``quantity of motion'' (as Newton called it), or
  what we call \define{Momentum}\index{Momentum|textbf}
  \begin{equation}
    \vec{p} = m\vec{v}
  \end{equation}
  Newton originally specified his second Law, in an inertial reference
  frame, as
  \begin{equation}
    \vec{F} = \frac{\D\vec{p}}{\D t}
  \end{equation}
  This version is used if $m$ is not a constant with respect to
  time. (For example, modeling a segway with someone stepping off.)
\end{remark}
\begin{remark}\index{Force!External}
The forces are only \emph{external} forces, exerted on the given body by
other external bodies in the environment. We do not consider forces
exerted by a body on itself, otherwise we get physically nonsensical
results.
\end{remark}

\N{No-Go on Higher Order Versions}
Suppose we considered something like
\begin{equation}
\vec{F} = \frac{\D\vec{p}}{\D t} + \alpha\frac{\D^{2}\vec{p}}{\D t^{2}}
\end{equation}
where $\alpha\approx 0.000327$. Could this even be a candidate for a
physical law?

No, this is not possible. There are a number of problems, all stem from
Ostragradski's theorem\index{Ostragradski's Theorem!higher derivative dynamics}. The basic problem with this: the system is
unstable, and the energy unbounded from below. (We would be able to
harness an infinite amount of energy from \emph{any} system if this
relation were true.)

\subsection{Newton's Third Law}
\begin{corollary}
From Newton's second and third laws, total linear momentum
is a conserved quantity.
\end{corollary}
\begin{proof}
To show something, $f$, is a conserved quantity, we
need to show its time derivative vanishes. For two interacting
bodies, the total momentum is
\begin{equation}
    \vec{p}_{\text{tot}} = \vec{p}_{1} + \vec{p}_{2}.
\end{equation}
Newton's third law tells us
\begin{equation}
    \vec{F}_{1}=-\vec{F}_{2}\implies\vec{F}_{1}+\vec{F}_{2}=0
\end{equation}
but using his second law, we find
\begin{equation}
    \frac{\D}{\D t}(\vec{p}_{1} + \vec{p}_{2}) =\vec{F}_{1}+\vec{F}_{2}=0
\end{equation}
which proves the claim.
\end{proof}