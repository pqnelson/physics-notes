\section{Kinematics}

\N{Overview}
When we consider \emph{how to describe the state of a system},
we are discussing the system's \define{Kinematics}. For classical
mechanics, we begin with a point-like particle which travels along a
curve called its \define{Trajectory}. 

At any moment, a point in the trajectory is the particle's
\emph{position}. The first derivative (with respect to time) is the
particle's \emph{velocity}. Its second derivative, its
\emph{acceleration}. Kinematics for a point particle amounts to little
more than the differential geometry of curves.

\subsection{Acceleration Vector}

\M We will consider the acceleration vector as the sum of two
vectors. Namely the \define{Parallel Acceleration} $\vec{a}_{\parallel}$
and the \define{Perpendicular Acceleration} $\vec{a}_{\perp}$. So we
decompose acceleration as
\begin{equation}
\vec{a} = \vec{a}_{\parallel}+\vec{a}_{\perp}
\end{equation}
in general. We summarize the components in the diagram:
\begin{center}
\includegraphics{img/kinematics.0}
\end{center}
\noindent%
The question we should ask ourselves is ``What happens to velocity (when
considering nonzero acceleration)?''

\N{Parallel Acceleration} 
So consider a ``small interval'' from $t_{1}$ to $t_{2}=t_{1}+\Delta t$. 
The change in velocity $\Delta\vec{v}=\vec{v}(t_{2})-\vec{v}(t_{1})$
gives us an ``average velocity'' for the body. If the acceleration is
parallel to the velocity, then $\Delta\vec{v}$ is in the same direction
as $\vec{v}_{1}=\vec{v}(t_{1})$. And the velocity's direction remains the same
(since $\Delta\vec{v}$ is in the same direction as $\vec{v}_{1}$). So
the magnitude for the velocity \emph{changes} when the acceleration is
\emph{parallel} to velocity.

\N{Perpendicular Acceleration}\label{chunk:perpAcceleration}
Suppose we have in our interval $\Delta t$, the change in velocity
$\Delta\vec{v}$ is ``very nearly'' perpendicular to $\vec{v}_{1}$. We
doodle the situation thus:
\begin{center}
\includegraphics{img/kinematics.1}
\end{center}
\noindent%
Again, we have $\Delta\vec{v}=\vec{v}_{2}-\vec{v}_{1}$, but in this case
$\vec{v}_{1}$ and $\vec{v}_{2}$ differ in directions and have the
\emph{same magnitude}. As we take $\Delta t\to0$, we see $\theta\to0$,
and $\Delta\vec{v}$ becomes perpendicular to \emph{both} $\vec{v}_1$ and
$\vec{v}_2$. So, in other words, the speed (magnitude of velocity)
remains the same but the direction of the paths change.

\N*{Summary}
When the magnitude of velocity changes but the path remains the same,
this is parallel acceleration. When the magnitude of velocity stays
constant while the path changes, this is perpendicular
acceleration. Furthermore
\begin{equation}
\sgn(\vec{v}\cdot\vec{a})=\begin{cases}
-1 & \mbox{if speed is decreasing}\\
0 & \mbox{if speed is constant}\\
+1 & \mbox{if speed is increasing}
\end{cases}
\end{equation}
This is because
\begin{equation}
\begin{split}
\frac{\D}{\D t}\|\vec{v}\|^{2}
&=\frac{\D}{\D t}(\vec{v}\cdot\vec{v})\\
&=2\vec{v}\cdot\frac{\D\vec{v}}{\D t}\\
&=2\vec{v}\cdot\vec{a}
\end{split}
\end{equation}
but by the product rule we also have this equal to
\begin{equation}
\frac{\D}{\D t}\|\vec{v}\|^{2}=2\|\vec{v}\|\cdot\frac{\D\|\vec{v}\|}{\D t}
\end{equation}
which has the same sign as $\D\|\vec{v}\|/\D t$ since
$\|\vec{v}\|\geq0$. 

\subsection{Projectile Motion}
\N{Definition}
A \define{Projectile} is any body that is given an initial velocity,
then follows a trajectory determined entirely by gravitational
acceleration and air resistance. The path taken by a projectile is
called its \define{Trajectory}.

\begin{rmk}
We simplify this model by considering constant
acceleration\footnote{Note this is the \emph{definition} of standard
  gravity, as set forth by ISO 80000-3. See~\cite[p 52]{sp330}.}
\begin{equation}
g=\SI{9.80655}{\meter/\square\second} 
\end{equation}
and neglecting air resistance. This limits us, since we ignore the
curvature of the Earth, but for a first pass, it's decent.
\end{rmk}

\M We set up coordinates so the horizontal coordinate $x$ experiences no
acceleration, and the vertical component $y$ experiences $g$. So we have
\begin{equation}
\vec{a} = (0,\; -g).
\end{equation}
We need to specify the initial position $\vec{x}(t_{0})=(x_{0},\; y_{0})$
for some initial time $t_{0}$ (usually taken to be zero). Then the
velocity vector is
\begin{equation}
\vec{v}(t) = (v_{0}\cos(\alpha),\; v_{0}\sin(\alpha) - gt)
\end{equation}
where the object has its initial velocity given with magnitude $v_{0}$
and angle relative to the horizontal $\alpha$. We can solve this using
calculus, and get the position along the trajectory as
\begin{equation}
\vec{x}(t) = \left(x_{0} + v_{0}\cos(\alpha)t,\; y_{0} + v_{0}\sin(\alpha)t-\frac{1}{2}gt^{2}\right).
\end{equation}
This is the first step in the right direction.

\N{Parabolic Motion}
Lets take $(x_{0}, y_{0})=(0,0)$ --- the initial position is the origin
of the coordinate system. We then have
\begin{subequations}
\begin{align}
&\left\{
\begin{array}{l}
x(t) = v_{0}\cos(\alpha)t\\
y(t) = v_{0}\sin(\alpha)t - gt^{2}/2
\end{array}\right.\\
&\left\{
\begin{array}{l}
v_{x}(t) = v_{0}\cos(\alpha)\\
v_{y}(t) = v_{0}\sin(\alpha) - gt
\end{array}\right.
\end{align}
\end{subequations}
Observe then that
\begin{equation}
t = \frac{x(t)}{v_{0}\cos(\alpha)}.
\end{equation}
We plug this into $y(t)$ to find $y$ as a function of $x$:
\begin{equation}
\begin{split}
y(x) 
&=v_{0}\sin(\alpha)\left(\frac{x(t)}{v_{0}\cos(\alpha)}\right)
   - \frac{1}{2}g\left(\frac{x(t)}{v_{0}\cos(\alpha)}\right)^{2}\\
&=\tan(\alpha)x - \frac{g}{2{v_{0}}^{2}}\sec^{2}(\alpha)x^{2}\\
&=Ax-Bx^{2}
\end{split}
\end{equation}
where $A=\tan(\alpha)$ and $B=g\sec^{2}(\alpha)/2{v_{0}}^{2}$ are
constants in time. Hence we get an honest parabola describing the
trajectory.

\begin{rmk}[Assumptions]
Air resistance isn't always negligible. For example, we will consider
the trajectory for a Napoleon 12-pounder canon. When we ignore air
resistance, we predict the canonball shot will travel twice farther
than its empirical distance.
\end{rmk}

\begin{rmk}[Historical Notes]
Galileo appears to be the first to note parabolic motion in the second
half of his \emph{Dialogues of the Two New Sciences} (1638). Not only
that, Galileo investigated the motion of bodies under constant
acceleration. Presumably this summarizes his earlier work dating back to
1604. Although it sounds minor, this work was crucial in overturning the
Aristotlean worldview that ``heavier'' objects fall faster than
``lighter'' objects. 
\end{rmk}

\workedExamples{}
%%%%%%%%%%%%%%%%%%%%%%%%%%%%%%%%%%%%%%%%

\N[Galileo and Tower of Pisa]{Example}
Suppose a student emulates Galileo and drops a coin\footnote{Galileo
  supposedly dropped two canon balls of different composition, but
  the guards won't let you take canonballs into the Tower of Pisa for
  ``security reasons''.} from the
leaning Tower of Pisa (at $\SI{56.0}{\meter}$ from the ground). How
long until the coin hits the ground? 

\begin{soln}
\textsc{Identify.} We need to use
$x(t)=x_{0}+v_{0}t+\frac{1}{2}at^{2}$, assuming $a=-g$ is
constant. 

\textsc{Set-Up.}
We first doodle the tower
\begin{center}
  \includegraphics{img/pisa.0}
\end{center}
We pick the origin at the top of the tower where the student
drops the coin at time $t=0$. So $y_{0}=56.0$, $v_{0}=0$ since the
coin is dropped, and $a=\SI{-9.8}{\meter\per\second\squared}$, and we
want\footnote{This was also chosen because if this problem were
  ``impossible to do by hand'', we can make the computer find the root
  of a function easier than we could make it find a fixed point. Really,
they're the same, but one works ``out of the box'' while the other
requires us to fiddle around with algebra.} to find $t_{1}$ such that $y(t_{1})=0.0$.

\textsc{Execute.}
We have
\begin{equation}
\begin{split}
y(t) &= y_{0} + v_{0}\sin(\alpha)t - \frac{1}{2}gt^{2}\\
&=y_{0} - \frac{1}{2}gt^{2}
\end{split}
\end{equation}
We see then that
\begin{equation}
y(t_{1}) = 0\implies t_{1}=\sqrt{\frac{2y_{0}}{g}}.
\end{equation}
We plug in our initial values to find
\begin{equation}
\begin{split}
t_{1} &= \sqrt{\frac{2\cdot\SI{56.0}{\meter}}{\SI{9.8}{\meter\per\second\squared}}}\\
&\approx \SI{3.4}{\second}
\end{split}
\end{equation}

\textsc{Evaluate.}
We see that the coin, after $\SI{3.0}{\second}$, travels a distance
of approximately $\Delta y\approx \SI{45}{\meter}$. Why? Well,
since $g/2\approx \SI{5}{\meter/\second\squared}$ and 
$(\SI{3}{\second})^{2}=\SI{9}{\second\squared}$ gives us $\Delta y\approx(1/2)g\cdot(\SI{9.0}{\second\squared})\approx \SI{45}{\meter}$. So it's
reasonable that it'd take about $\SI{3.4}{\second}$.
\end{soln}

%%%%%%%%%%%%%%%%%%%%%%%%%%%%%%%%%%%%%%%%
\makeatletter
\N[{Young and Freedman~\cite[Exercise 3.9]{young}}]{Example}
\makeatother
A book slides off a horizontal table top with a speed of
$\SI{1.10}{\meter/\second}$. It strikes the floor in
$\SI{0.35}{\second}$. Ignore air resistance.
\begin{enumerate}
\item Find the height of the table top above the floor.
\item Find the horizontal distance from the table's edge to where it
  hits the floor.
\end{enumerate}

\begin{soln}
\textsc{Identify and Set Up.}
We first doodle the situation
\begin{center}
\includegraphics{img/kinematics.2}
\end{center}
Let $t_{1}=\SI{0.35}{\second}$ be the time it takes for the book to hit
the ground. We see the book has initial height $h$, so $y(0)=h$ and
$y(t_{1})=0$. The initial velocity vector is moving in the $x$-direction
only, so $\vec{v}_{0}=\SI{1.10}{\meter\per\second}\hat{x}$. We abuse
notation and write $v_{0}=\SI{1.10}{\meter\per\second}$ for the speed. We use
projectile motion, so
\begin{equation*}
\begin{split}
y &= y_{0} - \frac{1}{2}gt^{2}\\
x &= x_{0} + v_{0}t.
\end{split}
\end{equation*}

\textsc{Execute.} (1) Given the time it takes for the book to fall on the
ground, we can algebraically solve for the height writing
\begin{equation}
y(t_{1})=0\implies y_{0}=\frac{1}{2}gt^{2}
\end{equation}
which numerically has the value
\begin{equation}
y_{0}\approx \SI{0.601}{\meter}.
\end{equation}
(2) We have the book's initial position be such that $x_{0}=0$. Then the
horizontal displacement becomes
\begin{equation}
\begin{split}
x(t_{1}) &= v_{0}t\\
&= \SI{0.385}{\meter}
\end{split}
\end{equation}

\textsc{Evaluate.}
We see for subproblem (1) as $t\to0$ we expect $y_{0}\to0$, which
happens with our algebraic solution. Furthermore, rearranging this
algebraically gives us the correct expression for the time until a body
hits the ground without any initial velocity:
$\sqrt{2y_{0}/g}=t_{1}$. So this is algebraically correct, we could have
only erred through arithmetic.

Similarly, for subproblem (2), as $v_{0}\to0$ we should recover the
previous problem's solution --- a free fall without any initial
velocity. Indeed, $x(t)\to x_0$ in this limit, which suggests we're on
the right track.
\end{soln}

%%%%%%%%%%%%%%%%%%%%%%%%%%%%%%%%%%%%%%%%

\N[Napoleon 12-Pounder]{Example}
Wikipedia tells us\footnote{Accessed April 13, 2014 at 2:31PM (PST), \url{http://en.wikipedia.org/wiki/Canon_obusier_de_12}}
that the Napoleon 12-pounder canon had a muzzle velocity of
\SI{439}{\meter/\second} (i.e., the velocity which a canonball had
measured when it left the muzzle). What is the range at \ang{5} elevation?

\begin{soln}
\textsc{Identify and Setup.} We see this is a projectile motion
problem. We're given $\alpha=\ang{5}$, and
$v_{0}=\SI{439}{\meter/\second}$. We estimate the muzzle is
$\SI{1}{\meter}$ from the ground. So we pick coordinates such that the
horizontal axis is the $x$ coordinate, and the vertical axis is the $y$
coordinate. The initial position would be
\begin{equation}
\vec{x}(0)=(0,\SI{1}{\meter}).
\end{equation}
We use the projectile equations
\begin{equation*}
\begin{split}
x(t) &= x_{0} + v_{0}\cos(\alpha)t\\
y(t) &= y_{0} + v_{0}\sin(\alpha)t - \frac{1}{2}gt^{2}
\end{split}
\end{equation*}
This gives us enough to start plugging away.

\textsc{Execute.}
We want to consider the value of $x(t)$ at the time $t_{1}$ when the
canonball has hit the ground $y(t_{1})=0$. We see that $y(t)=0$ when
\begin{equation}
t_{\mp} = \frac{1}{g}\left(v_{0}\sin(\alpha)
         \mp\sqrt{v_{0}^{2}\sin^{2}(\alpha) + 2y_{0}g}\right)
\end{equation}
We're interested in the time $t>0$, so this gives us
\begin{equation}
t_{1} = \frac{1}{g}\left(v_{0}\sin(\alpha)
         + \sqrt{v_{0}^{2}\sin^{2}(\alpha) + 2y_{0}g}\right)
\end{equation}
So, we see
\begin{equation}
x(t_{1}) = \frac{v_{0}\cos(\alpha)}{g}\left(v_{0}\sin(\alpha)
         + \sqrt{v_{0}^{2}\sin^{2}(\alpha) + 2y_{0}g}\right)
\end{equation}
When we plug in our initial conditions, we get
\begin{subequations}
\begin{equation}
t_{1}=\SI{7.83}{\second}
\end{equation}
and
\begin{equation}
x(t_{1})=\SI{3424}{\meter}.
\end{equation}
\end{subequations}

\textsc{Evaluate.} Was this reasonable? Well, we see that for $t_{1}$,
we can estimate it as
\begin{equation}
t_{1}\approx \frac{2v_{0}\sin(\alpha)}{10}
\end{equation}
since $g\approx 10$, $\cos(\alpha)\approx 1$, and the
$\sqrt{v_{0}^{2}\sin^{2}(\alpha)+\dots}\approx 
v_{0}\sin(\alpha)$. We see then that
\begin{equation}
t_{1}\approx \SI{7.6}{\second}.
\end{equation}
We plug this back into $x(t)$ to find
\begin{equation}
x(\SI{7.6}{\second})\approx v_{0}t = \SI{3336.4}{\meter}.
\end{equation}
This rough order-of-magnitude estimate confirms our calculations were
correct\dots probably.
\end{soln}

\begin{rmk}
Note the range for a Napoleon 12-pounder with the same given conditions
is empirically around \SI{1480}{\meter}. We computed over twice
that. Why? We ignored air resistance. For discussions, see
Burko and Price~\cite{burko2003}, Euler~\cite{e77,e217,e226,e853}, 
Grimberg \emph{et al.}~\cite{grimberg2008}, and Hogg~\cite{hogg2007}. 
\end{rmk}

\subsection{Uniform Circular Motion}

\N{Definition}
When a particle moves in a circular trajectory with constant speed, the
motion is called \define{Uniform Circular Motion}.

\N{Derivation of Acceleration}
We draw a segment of the trajectory. The particle mvoes about
$\mathcal{O}$ with radius $R$. We have the velocities $\vec{v}_{1}$ at
$P_{1}$, and $\vec{v}_{2}$ at $P_{2}$ some time $\Delta t$ later. The
distance $\Delta\vec{s}=\overrightarrow{P_{1}P_{2}}$. This forms a triangle, as
doodled, with angle $\Delta\theta$:
\begin{center}
\includegraphics{img/kinematics.3}
\end{center}
With the velocity vectors, we parallel transport $\vec{v}_{1}$ to the
base point of $\vec{v}_{2}$, thus constructing the following triangle:
\begin{center}
\includegraphics{img/kinematics.4}
\end{center}
We claim that $\Delta\varphi=\Delta\theta$ since $\vec{v}_{1}$ is
tangent to the circle, as is $\vec{v}_{2}$ (extend
$\overrightarrow{OP}_{2}$ to infinity, parallel transport $\vec{v}_{2}$
along it, extend $\vec{v}_{1}$ to infinity, and parallel transport the
vector $\vec{v}_{1}$ along that ray --- the rest is high school geometry).

\N*{Lemma} The triangles in both these figures are similar.

\begin{proof}
By the side-angle-side theorem, it follows immediately.
\end{proof}

Since these two triangles are similar, we have
\begin{equation}
\frac{\|\Delta\vec{v}\|}{v_{1}}=\frac{\Delta s}{R}
\end{equation}
where dropping the arrow indicates we are considering magnitudes, this
implies
\begin{equation}
\|\Delta\vec{v}\| = \frac{v_{1}}{R}\Delta s
\end{equation}
The average acceleration during this period is then
\begin{equation}
\begin{split}
a_{\text{avg}}&=\frac{\|\Delta\vec{v}\|}{\Delta t}\\
&=\frac{v_{1}}{R}\frac{\Delta s}{\Delta t}
\end{split}
\end{equation}
The instantaneous acceleration we recover taking $P_{2}\to P_{1}$ (or,
equivalently, as $\Delta t\to0$):
\begin{equation}
\begin{split}
a_{\text{inst}} &= \lim_{\Delta t\to0}\frac{v_{1}}{R}\frac{\Delta s}{\Delta t}\\
&=\frac{v_{1}}{R}\lim_{\Delta t\to0}\frac{\Delta s}{\Delta t}
\end{split}
\end{equation}
Observe the limit $\Delta s/\Delta t\to v_{1}$ as $P_{2}\to P_{1}$. We
let $P_{1}$ be arbitrary, so we can drop the subscript and conclude
\begin{equation}
a = \frac{v^{2}}{R}\quad\mbox{for uniform circular motion}.
\end{equation}

\N{Centripetal Acceleration}
Because speed is constant, the acceleration vector is orthogonal to the
velocity vector (\S\ref{chunk:perpAcceleration}). Which way does
the acceleration vector point? The acceleration vector points towards
the concave side of the trajectory, i.e., towards the center of the
circle.

In uniform circular motion, the magnitude $a$ of the (instantaneous)
acceleration is equal to the square of the speed $v$ divided by the
radius. Its direction is perpendicular to $\vec{v}$ and points towards
the circle's origin.

\N{Definitions}
The acceleration vector for uniform circular motion always points
towards the circle's center, and is called \define{Centripetal Acceleration} 
where ``centripetal'' derives from the Latin for ``center
[\emph{centrum}] seeking [\emph{-petus}]''.

We express the magnitude of acceleration in terms of the \define{Period}
$T$ of motion (time for one revolution about the circle). We see
\begin{subequations}
\begin{align}
v &= \frac{2\pi R}{T}\\
a &=\frac{4\pi^{2}R}{T^{2}}.
\end{align}
\end{subequations}

\begin{rmk}[Historical Notes]
The tradition to introduce uniform circular motion when discussing
kinematics may be traced back to chapter 1, section 2 of
Book I in Isaac Newton's \emph{Principia}. Newton, too, coined the term
``centripetal'' around 1687.
\end{rmk}

\workedExamples{}
%%%%%%%%%%%%%%%%%%%%%%%%%%%%%%%%%%%%%%%%

\N[{Earth Rotation~\cite[Exercise 3.25]{young}}]{Example}
The Earth rotates in a period of 24 hours, its radius is
$\SI{6371}{\kilo\meter}$. 
\begin{enumerate}
\item Find the radial acceleration $a_{\text{rad}}$ of the Earth (on the
  equator).
\item If $a_{\text{rad}}>g$, then objects will fly off the Earth. What
  would the period be for this to happen?
\end{enumerate}

\begin{soln}
\textsc{Identify and Set-up.}
We see this is just finding the radial acceleration
\begin{equation*}
a = \frac{4\pi^{2}R}{T^{2}}
\end{equation*}
then considering the period to satisfy
\begin{equation}
T^{2}\leq \frac{4\pi^{2}R}{g} \implies
T\leq2\pi\sqrt{\frac{R}{g}}.
\end{equation}
We are given
\begin{equation}
R_{\oplus} = \SI{6371}{\kilo\meter}
\end{equation}
and
\begin{equation}
T_{\oplus} = \SI{24}{\hour} = \SI{86400}{\second}
\end{equation}
where the subscript $\oplus$ is the symbol for the Earth. We just have
to plug these into the given equations to get our solutions.

\textsc{Execute.}
(1) We find 
\begin{equation}
\begin{split}
a &= \frac{4\pi^{2}\times\SI{6.371}{\meter}}{(\SI{86.4}{\second})^{2}}\\
&= \SI{0.03369}{\meter\per\second\squared} = 3.436\times10^{-3}g.
\end{split}
\end{equation}
(2) We similarly find
\begin{equation}\label{eq:approxPeriodForEarthToThrowEverythingoffSurface}
T\leq\SI{5064}{\second}=\SI{1.407}{\hour}.
\end{equation}

\textsc{Evaluate.} We see that
\begin{equation}
a=\frac{4\pi^{2}R_{\oplus}}{T^{2}}\orderOf3\times10^{-3}g
\end{equation}
and we wanted to find a period, say, $U$ such that
\begin{equation}
a_{\text{rad}} = \frac{4\pi^{2}R_{\oplus}}{U^{2}}\orderOf g
\end{equation}
We take the ratio
\begin{equation}
\frac{a}{a_{\text{rad}}}=\frac{U^{2}}{T^{2}}\orderOf3\times10^{-3}
\end{equation}
hence
\begin{equation}
U\orderOf T\times\sqrt{30}\times10^{-2} \approx 4.7\times10^{3}\si{\second}
\end{equation}
which is the correct order of magnitude as Eq \eqref{eq:approxPeriodForEarthToThrowEverythingoffSurface}.
\end{soln}

\subsection{Non-Uniform Circular Motion}

\subsection{Relative Motion}
