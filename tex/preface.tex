
\M
These are my cumulative notes on physics. They are mildly pedagogical,
in some sense. I'd like to review some of my underlying philosophy.

Ultimately, science is a collection of fields which predicts the state
of the world. It's organized into distinct fields for simplicity. We
review physics here. For each field, there are different ``intellectual
toolkits'' or \define{Paradigms}\footnote{Technically, I suppose this is
closer to Lakatos' outlook rather than Kuhn's, but since the term
``Paradigm'' is so widespread\dots I figured I might as well use it.}
to explain phenomena and make predictions.

\M A paradigm consists of a ``hard core'' of equations, etc. There's a
``soft underbelly'' of ancillary tools compatible with (but in no way
imply any of) the ``hard core''. We will see that Newton's Laws are the
``core'' of classical mechanics, and aether is one possible ``soft''
theory.

Paradigms have a specific scope where it works. It makes sense to
discuss Newtonian mechanics for slow, big objects. But when we get to
the atomic scale, it no longer makes effective predictions. When a
paradigm cannot effectively predict, we must look for alternatives which
--- when we restrict focus to a particular scale --- recovers the old
toolkit.

\M Science boils down to studying examples. Physics has a unique
property: we can vary the parameters for the solution to recover a
simpler problem's solution. Consider someone jumping off a cliff --- if
we solve it algebraically, the solution \emph{should} depend on the
initial velocity. And if we take the limit as that initial velocity goes
to zero (``vanishes''), we should recover the free-fall scenario. This
is great because free fall is easy to calculate. When we add more
conditions, it's not so obvious what the solution should be. So we can
double check our work by taking appropriate limits.

But paradigms have this ``limiting procedure'' to recover the physics we
know and love. If we couldn't explain everyday phenomena with quantum
theory, then obviously something is very wrong. Fortunately, we
can\footnote{Well, kind of. More on this later!}. So really science
consists of successive refinements explaining --- everything.

\M A word about the writing style. It's completely
idiosyncratic. The style emulates mathematics. We have a collection of
definitions, theorems, and examples. Physics focuses on the latter. For
better or worse, Euler and Euclid has influenced the writing style. It
seems to work well when writing notes for one's self.

If one were to study the writing of physics, one would realize indeed it
boils down to a grocery list of examples\dots until graduate
school. Consequently, formalizing the structure of an example seems
advantageous. Young and Freedman~\cite{young} have done this to a degree. We
attempt to follow it throughout as best as we can.

One of the interesting points of departure between mathematics and
science is the role of examples. For mathematicians, it's to show what a
property looks like manifested in a mathematical object (or how it's
\emph{not} manifested). Or to show an application of an algorithm. But
for physicists, their example repertoire serves to rule out explanations
and provide guidance with their reasoning. Examples for scientists are
like entrees for chefs.

\vfil
\begin{quotes}
Science is organised knowledge; and before knowledge can be organised,
some of it must be possessed. Every study, therefore, should have a
purely experimental introduction; and only after an ample fund of
observations has been accumulated, should reasoning begin.
\author Herbert Spenser, \textsl{Essays on Education} (1861)

\

Science is built up with facts, as a house is with stones.
But a collection of facts is no more a science
than a heap of stones is a house.
\author Henri Poincar\'e, \textsl{Science and Hypothesis} (1901)

\

But beyond the bright searchlights of science,
Out of sight of the windows of sense,
Old riddles still bid us defiance,
Old questions of Why and of Whence.
\author W.\ C.\ D.\ Whethem %, \textsl{Recent Development of Physical Science}
(1904)
\end{quotes}
