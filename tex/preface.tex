\begin{quotes}
In learning the sciences
examples are of more use than precepts.
\author Isaac Newton, \textsl{Arithmetica Universalis} (1707)
\end{quotes}

\M
These are my cumulative notes on physics. They are mildly pedagogical,
in some sense. I'd like to review some of my underlying philosophy.

Ultimately, science is a collection of fields which predicts the state
of the world. It's organized into distinct fields for simplicity. We
review physics here. For each field, there are different ``intellectual
toolkits'' or \define{Paradigms}\footnote{Technically, I suppose this is
closer to Lakatos' outlook rather than Kuhn's, but since the term
``Paradigm'' is so widespread\dots I figured I might as well use it.}
to explain phenomena and make predictions.

\M A paradigm consists of a ``hard core'' of equations, etc. There's a
``soft underbelly'' of ancillary tools compatible with (but in no way
imply any of) the ``hard core''. We will see that Newton's Laws are the
``core'' of classical mechanics, and aether is one possible ``soft''
theory.

Paradigms have a specific scope where it works. It makes sense to
discuss Newtonian mechanics for slow, big objects. But when we get to
the atomic scale, it no longer makes effective predictions. When a
paradigm cannot effectively predict, we must look for alternatives which
--- when we restrict focus to a particular scale --- recovers the old
toolkit.

\N{Examples} Science boils down to studying examples. I quote Newton's
statement frequently that, ``In learning the sciences examples are of
more use than precepts,'' because it's true. When learning physics, it's
worth spending considerable time studying examples of Newton's second
Law, rather than enumerating the different types of forces (or anything
else). In fact, there's precious little room for ``proofs'' as in
mathematics, because we don't really have ``first principles'' in the
same way. Arguably, there's some ``undefined notions'' (like a ``point''
in Euclidean geometry) which is given and primitive, and we learn how to
work with it; in mechanics, this would be Newton's Laws of
motion \emph{or} the principle of stationary action (depending on
whether one is learning Newtonian or analytical mechanics). We
understand them through examples in physics, not theorems.

There is difficulty finding the ``sweet spot'' when writing notes about
physics. If we show too little work, then it is of little use to people
who do not already know the material. If we show too much work, then we
lose the forest for the trees.


\N{``Evaluate'' Step}
Physics has a unique property: we can vary the parameters for the
solution to recover a simpler problem's solution. Consider someone
jumping off a cliff --- if we solve it algebraically, the
solution \emph{should} depend on the initial velocity. And if we take
the limit as that initial velocity goes to zero (``vanishes''), we
should recover the free-fall scenario. This is great because free fall
is easy to calculate. When we add more conditions, it's not so obvious
what the solution should be. So we can double check our work by taking
appropriate limits.

But paradigms have this ``limiting procedure'' to recover the physics we
know and love. If we couldn't explain everyday phenomena with quantum
theory, then obviously something is very wrong. Fortunately, we
can\footnote{Well, kind of. More on this later!}. So really science
consists of successive refinements explaining --- everything.

\N{Style} A word about the writing style. It's completely
idiosyncratic. The style emulates mathematics. We have a collection of
definitions, theorems, and examples. Physics focuses on the latter. For
better or worse, Euler and Euclid has influenced the writing style. It
seems to work well when writing notes for one's self.

If one were to study the writing of physics, one would realize indeed it
boils down to a grocery list of examples\dots until graduate
school. Consequently, formalizing the structure of an example seems
advantageous. Young and Freedman~\cite{young} have done this to a degree. We
attempt to follow it throughout as best as we can.

One of the interesting points of departure between mathematics and
science is the role of examples. For mathematicians, it's to show what a
property looks like manifested in a mathematical object (or how it's
\emph{not} manifested). Or to show an application of an algorithm. But
for physicists, their example repertoire serves to rule out explanations
and provide guidance with their reasoning. Examples for scientists are
like entrees for chefs.

\vfil
\begin{quotes}
In learning the sciences
examples are of more use than precepts.
\author Isaac Newton, \textsl{Arithmetica Universalis} (1707)

\

Science is organised knowledge; and before knowledge can be organised,
some of it must be possessed. Every study, therefore, should have a
purely experimental introduction; and only after an ample fund of
observations has been accumulated, should reasoning begin.
\author Herbert Spenser, \textsl{Essays on Education} (1861)

\

Science is built up with facts, as a house is with stones.
But a collection of facts is no more a science
than a heap of stones is a house.
\author Henri Poincar\'e, \textsl{Science and Hypothesis} (1901)

\

But beyond the bright searchlights of science,
Out of sight of the windows of sense,
Old riddles still bid us defiance,
Old questions of Why and of Whence.
\author W.C.D.\ Whethem, \textsl{Recent Development of Physical Science} (1904)
\end{quotes}
