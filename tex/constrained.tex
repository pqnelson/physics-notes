\section{Gauge Invariance, Constraints}
\M
For a gauge system, the initial conditions alone do not uniquely
determine the state in the future. The general solutions to the
equations of motion may contain arbitrary functions of time.

We will see the canonical variables are not all independent. Hence a
gauge system is always a constrained system, but the converse is not
always true.

\subsection{The Lagrangian as a Starting Point. Primary Constraints}
\N{Stationary Action}
The classical equations of motion are those that make the action
\begin{equation}
  \action_{L} = \int^{t_{2}}_{t_{1}} L(q,\dot{q})\,\D t
\end{equation}
stationary under variations $\delta q^{n}(t)$ of the Lagrangian
variables $q^n$ (for $n=1,\dots,N$) such that the variation vanishes at
the endpoints $\delta q^{n}(t_{1})=\delta q^{n}(t_{2})=0$.

\N{Equations of Motion}
The conditions for the action to be stationary are preceisely the
Euler-Lagrange equations
\begin{equation}
  \frac{\D}{\D t}\left(\frac{\partial L}{\partial\dot{q}^{n}}\right)
  -\frac{\partial L}{\partial q^{n}}=0
\end{equation}
for $n=1,\dots, N$. We may use the chain rule to rewrite
\begin{equation}
  \frac{\D}{\D t}\left(\frac{\partial L}{\partial\dot{q}^{n}}\right)
  =\frac{\D q^{n'}}{\D t}\frac{\partial^{2} L}{\partial q^{n'}\partial\dot{q}^{n}}
  +\frac{\D\dot{q}^{n'}}{\D t}\frac{\partial^{2} L}{\partial\dot{q}^{n'}\partial\dot{q}^{n}}
\end{equation}
where we use Einstein summation convention. The equations of motion then
become
\begin{equation}
  \ddot{q}^{n'}\left(\frac{\partial^{2}L}{\partial\dot{q}^{n'}\partial\dot{q}^{n}}\right)
  =\frac{\partial L}{\partial q^{n}}
  -\frac{\partial^{2}L}{\partial q^{n'}\partial\dot{q}^{n}}\dot{q}^{n'}
\end{equation}
Hence the accelerations are uniquely determined by position and
velocities at that time if and only if the matrix
$(\partial^{2}L/\partial\dot{q}^{m}\partial\dot{q}^{n})$ is invertible,
i.e., $\det(\partial^{2}L/\partial\dot{q}^{m}\partial\dot{q}^{n})\neq0$.

\M
On the other hand, if
$\det(\partial^{2}L/\partial\dot{q}^{n'}\partial\dot{q}^{n})=0$, then the
accelerations are not uniquely determined by positions and
velocities. Hence the solutions would contain aritrary functions of
time. The case we're interested in is precisely when this matrix cannot
be inverted.

\N{Canonical Momenta, Primary Constraints}
When starting the Hamiltonian analysis, we define the \define{Canonical Momenta}
by
\begin{equation}
  p_{n}=\frac{\partial L}{\partial\dot{q}^{n}}.
\end{equation}
Then the determinant in question is precisely
\begin{equation}
  \det\left(\frac{\partial p_{n'}}{\partial\dot{q}^{n}}\right)=0
\end{equation}
which says the Hessian matrix for the change of coordinates
$(q,\dot{q})\to(q,p)$ is not invertible.

What does this mean? Well, not all the momenta are independent of each
other. So we have some relationships among them
\begin{equation}
  \phi_{m}(q,p)=0
\end{equation}
for $m=1,\dots,M$, which follow just from the definition of the
momenta. The conditions $\phi_{m}(q,p)=0$ are called \define{Primary
  Constraints} to emphasize the equations of motion \emph{are not} used
to obtain these relations, and they imply no restriction on the
coordinates $q^{n}$ and their velocities $\dot{q}^{n}$.

\N{Definition}
The \define{Primary Constraint Surface} consists of the submanifold
smoothly embedded in phase space according to $\phi_{m}(q,p)=0$.

\N*{Remarks}
We have some simplifying assumptions and remarks worth carrying around.
\begin{enumerate}
\item There are $M'$ independent equations among the primary
  constraints, where $0\leq M'\leq M$.
\item We assume for simplicity the rank of
  $\partial^{2}L/\partial\dot{q}^{n}\partial\dot{q}^{n'}$ is $N-M'$
  (i.e., constant throughout phase space).
\item The primary constraint surface is a phase-space submanifold of
  dimension $2N-M'$.
\end{enumerate}

\M
It follows from the primary constraints that the inverse transformation
from the $p$'s to the $q$'s is multi-valued. Given a point $(q^{n}, p_{n})$
which satisfy the primary constraints, the inverse image $(q^{n}, \dot{q}^{n})$
that solves
\begin{equation}
  p_{n}(q,\dot{q}) = \frac{\partial L}{\partial\dot{q}^{n}}
\end{equation}
is not unique\dots since $\partial L/\partial q$ maps the
$2N$-dimensional manifold of the $q$'s and $\dot{q}$'s to the smaller
manifold of dimension $(2N-M')$.

% \includegraphics{img/constrained.0}

To render this transformation invertible, we must introduce extra
parameters (at least $M'$ in number) to indicate the location of
$\dot{q}$ on the inverse manifold. These parameters will appear as
Lagrange multipliers when we define the Hamiltonian and study its
properties.

\subsection{Conditions on the Constraint Functions}

\N{Motivating Example}\label{n:regularity-conditions:motivating-example}
Given some surface, there are different ways to represent it as the
zeroes of some equations. For example $p_{1}=0$ may be written as
$(p_{1})^{2}=0$ or $\sqrt{|p_{1}|}=0$, or even redundantly as (say)
$(p_{1})^{2}=0$ and $\sqrt{|p_{1}|}=0$.

Before continuing our Hamiltonian analysis, we must impose some
conditions (restrictions) on the choice of the functions $\phi_{m}$
representing the primary constraint surface. These are precisely the
\emph{Regularity Conditions}.

\N{Regularity Conditions}\label{n:constrained-hamiltonian:regularity-cond}
The $(2N-M')$-dimensional constraint surface $\phi_{m}=0$ should be
coverable by open regions, on each of which (``locally'') the
constraint functions $\phi_{m}$ are either
\begin{enumerate}
  \item ``independent'' $\phi_{m'}=0$ for $m'=1$, \dots, $M'$ such that
    the Jacobian matrix $\partial(\phi_{m'})/\partial(q^{n},p_{n})$ is
    rank $M'$ on the constraint surface
  \item ``dependent'' $\phi_{\bar{m}'}=0$ (for $\bar{m}'=M'+1$, \dots,
    $M$) which holds as a consequence of others ($\phi_{m'}=0\implies\phi_{\bar{m}'}=0$).
\end{enumerate}

\begin{thm}
The regularity condition on the Jacobian matrix
$\partial(\phi_{m'})/\partial(q^{n},p_{n})$ may be reformulated as any
of the following equivalent conditions:
\begin{enumerate}
\item\label{thm:regularity-cond:regular-coords}
  The functions $\phi_{m'}$ can be locally taken as the first $M'$
  coordinates of a new, regular coordinate system in the vicinity of the
  constraint surface.
\item\label{thm:regularity-cond:locally-linear-independent}
  The gradients $\D\phi_{1}$, \dots, $\D\phi_{M'}$ are locally
  linearly independent on the constraint surface, i.e.,
  $\D\phi_{1}\wedge\cdots\wedge\D\phi_{M'}\neq0$ (``zero is a regular
  value of the mapping defined by $\phi_{1}$, \dots, $\phi_{M'}$'').
\item\label{thm:regularity-cond:dirac-cond}
  The variations $\delta\phi_{m'}$ are of order $\varepsilon$ for
  arbitrary variations $\delta q^{j}$ and $\delta p_{j}$ of order
  $\varepsilon$.
\end{enumerate}
\end{thm}
\begin{proof}[Sketch of Proof]
We see independent constraints $\phi_{m'}$ $\iff$ \eqref{thm:regularity-cond:regular-coords}
immediately.

We see \eqref{thm:regularity-cond:regular-coords} $\iff$ \eqref{thm:regularity-cond:locally-linear-independent}
immediately.

Lastly, we find linearity condition \eqref{thm:regularity-cond:locally-linear-independent} $\iff$ \eqref{thm:regularity-cond:dirac-cond}
immediately.
\end{proof}

\N{Example}
Consider our motivating example \Mref{n:regularity-conditions:motivating-example}.
We see $p_{1}=0$ is admissable.

We see ``$p_{1}=0$ and $(p_{1})^{2}=0$'' is admissible. It is
regular. But look, it is redundant since $p_{1}=0\implies``p_{1}=0%
\mbox{\ and\ } (p_{1})^{2}=0"$. So it is a redundant, or ``dependent'',
constraint.

We find for $(p_{1})^{2}=0$ the Jacobian
$\partial({p_{1}}^{2})/\partial(q^{n}, p_{n})=0$ when
$(p_{1})^{2}=0$. That is, it is not rank-1 on the constraint
surface. Hence this constraint cannot be regular.

We find for $\sqrt{|p_{1}|}$ the Jacobian
$\partial(\sqrt{|p_{1}|})/\partial(q^{n},p_{n})$ is singular on the
constraint surface. This cannot possibly be regular.

\N{Remark}
Although we assumed this partition of the constraint functions may be
done locally, it is \emph{not} necessary to explicitly divvy them up
into dependent and independent constraints to develop our theory. The
subsequent formulas won't be based on such a split. All we require is
picking $\phi_{m}$ such that \emph{in principle} this could be done.

At any rate, when $\phi_{m}$ satisfy the regularity conditions, we have
two useful properties.

\begin{thm}
If a (smooth) phase space function $G$ vanishes on the surface
$\phi_{m}=0$, then $G=g^{m}\phi_{m}$ for some functions $g^{m}$.
\end{thm}

\begin{proof}
We use the fact that we may write a new regular coordinate system
$(y_{m'},x_{n})$ where $y_{m'}=\phi_{m'}$. Then we just note we want to
consider $G(0,x_{n})=0$. Using calculus, we see
\begin{subequations}
\begin{align}
G(y,x) &= \int^{1}_{0}\frac{\D}{\D t}G(ty, x)\,\D t\\    
&=\int^{1}_{0}y_{m'}\frac{\partial}{\partial y_{m'}}G(ty, x)\,\D t\\
&=y_{m'}\int^{1}_{0}\frac{\partial}{\partial y_{m'}}G(ty, x)\,\D t
\end{align}
\end{subequations}
Hence we find
\begin{equation*}
g^{m}=\int^{1}_{0}\frac{\partial}{\partial y_{m'}}G(ty, x)\,\D t.\qedhere
\end{equation*}
\end{proof}

\begin{thm}\label{thm:variation-tangent-to-constraint-surface}
If $\lambda_{n}\delta q^{n} + \mu^{n}\delta p_{n}=0$ for arbitrary
variations $\delta q^{n}$, $\delta p_{n}$ tangent to the constraint
surface, then
\begin{equation}
  \begin{split}
    \lambda_{n} &= u^{m}\frac{\partial\phi_{m}}{\partial q^{n}}\\
    \mu^{n} &= u^{m}\frac{\partial\phi_{m}}{\partial p_{n}}
  \end{split}
\end{equation}
for some $u^{m}$. (The equalities here are equalities on the constraint
surface.) 
\end{thm}
\begin{proof}
By the regularity conditions, the gradients $(\partial\phi_{m'}/\partial
q^{n},\partial\phi_{m'}/\partial p_{n})$ are linearly independent.
We use Artin's trick\footnote{See Artin~\cite{artin1991}, specifically
chapter 8 ``Linear Groups'', section 6 ``The Lie Algebra'' where Artin
discusses tangent vectors using infinitesimals.} for figuring out the
tangent vectors to the surface
\begin{equation}
f(q,p) = u^{m}\phi_{m}(q,p)\weakEq 0
\end{equation}
by writing
\begin{subequations}
\begin{align}
f(q+\delta q, p+\delta p) &= u^{m}\phi_{m}(q+\delta q, p+\delta p)\\
&=u^{m}\left(\phi_{m}(q,p)
             + \frac{\partial\phi_{m}(q,p)}{\partial q^{n}}\delta q^{n}
             + \frac{\partial\phi_{m}(q,p)}{\partial p_{n}}\delta p_{n}\right)\\
&=0\mbox{ by hypothesis}
\end{align}
\end{subequations}
then
\begin{equation}
\lambda_{n}\delta q^{n}+\mu^{n}\delta p_{n}
=u^{m}\left(\frac{\partial\phi_{m}(q,p)}{\partial q^{n}}\delta q^{n}
            + \frac{\partial\phi_{m}(q,p)}{\partial p_{n}}\delta p_{n}\right)
\end{equation}
which implies the desired result by matching coefficients of the variations.
\end{proof}


\subsection{The Canonical Hamiltonian}
\begin{defn}[Canonical Hamiltonian]
We introduce the Hamiltonian via the Legendre transformation, writing
the \define{Canonical Hamiltonian}
\begin{equation}
H = \dot{q}^{n}p_{n} - L.
\end{equation}
\end{defn}

\N{Canonical Hamiltonian Depends on Momenta, Positions}
As it is written, since $p=p(q,\dot{q})$, the Hamiltonian is a function
of the positions and velocities. We claim that velocities enter only
through the combinations of
\begin{equation}
  p_{n}(q,\dot{q}) = \frac{\partial L}{\partial\dot{q}^{n}}.
\end{equation}
How to check this claim?

We can verify this by considering $\delta H$, and seeing it is
proportional to $\delta p_{n}$ and $\delta q^{n}$. Observe
\begin{subequations}
\begin{align}
\delta H &= \delta(\dot{q}^{n}p_{n} - L)\\
&=\dot{q}^{n}p_{n}+p_{n}\dot{q}^{n}
  -\left(\frac{\partial L}{\partial q^{n}}\delta q^{n}
         + \frac{\partial L}{\partial\dot{q}^{n}}\delta\dot{q}^{n}\right)\\
&=\dot{q}^{n}p_{n}
   +\left(p_{n}-\frac{\partial L}{\partial\dot{q}^{n}}\right)\dot{q}^{n}
  -\frac{\partial L}{\partial q^{n}}\delta q^{n}\\
&=\dot{q}^{n}p_{n} -\frac{\partial L}{\partial q^{n}}\delta q^{n}.
\end{align}
\end{subequations}
Observe here $\delta p_{n}$ is \emph{not} an independent variation, but
regarded as a linear combination of $\delta q^{n}$ and
$\delta\dot{q}^{n}$. We see that this is the only way for
$\delta\dot{q}$'s to enter, implying $H=H(q,p)$.

\N{Non-Uniqueness}
We see the Hamiltonian defined by the Legendre transformation is not
necessarily unique\dots well, it's not uniquely determined as a function
of $p$'s and $q$'s. Why? Because the $\delta p$'s are not all
independent but are restricted to preserve the primary constraints
$\delta\phi_{m}\weakEq0$.

We conclude the canonical Hamiltonian is well-defined on the submanifold
defined by the primary constraints and can be extended arbitrarily off
that manifold. The formalism shouldn't change under the replacement
\begin{equation}
  H\to H+c^{m}(q,p)\phi_{m}
\end{equation}
for arbitrary (smooth) phase-space functions $c^{m}(q,p)$.

\N{Sketches of Hamilton's Equations}\label{n:constrained:eom-in-hamiltons-form}
We can rewrite the variation of the Hamiltonian as
\begin{subequations}
\begin{equation}
\delta H - \left(\dot{q}^{n}p_{n} -\frac{\partial L}{\partial q^{n}}\delta q^{n}\right) = 0
\end{equation}
then expand $\delta H$ as a linear combination of $\delta q$'s and
$\delta p$'s
\begin{equation}
\left(\frac{\partial H}{\partial q^{n}}\delta q^{n}
      +\frac{\partial H}{\partial p_{n}}\delta p_{n}\right)
- \dot{q}^{n}p_{n} + \frac{\partial L}{\partial q^{n}}\delta q^{n}=0
\end{equation}
then gathering terms
\begin{equation}
 \left(\frac{\partial H}{\partial p_{n}} - \dot{q}^{n}\right)\delta p_{n}
+\left(\frac{\partial H}{\partial q^{n}} + \frac{\partial L}{\partial q^{n}}\right)\delta q^{n}
=0
\end{equation}
\end{subequations}
Invoking theorem \ref{thm:variation-tangent-to-constraint-surface} we
find
\begin{subequations}
  \begin{align}
\delta q^{n}\colon\quad-\dot{q}^{n}&=-\frac{\partial H}{\partial p_{n}}
                                     +u^{m}\frac{\partial\phi_{m}}{\partial p_{n}}\\
\delta p_{n}\colon\quad\left.\frac{\partial L}{\partial q^{n}}\right|_{\dot{q}}
&=-\left.\frac{\partial H}{\partial q^{n}}\right|_{p}
   +u^{m}\frac{\partial\phi_{m}}{\partial q^{n}}
  \end{align}
\end{subequations}
Multiplying through by $-1$ and relabeling $u^{m}\to-u^{m}$ gives us
\begin{subequations}
\begin{align}
\dot{q}^{n}&=\frac{\partial H}{\partial p_{n}} + u^{m}\frac{\partial\phi_{m}}{\partial p_{n}}\\
\left.-\frac{\partial L}{\partial q^{n}}\right|_{\dot{q}}
&=\left.\frac{\partial H}{\partial q^{n}}\right|_{p}
   +u^{m}\frac{\partial\phi_{m}}{\partial q^{n}}
\end{align}
\end{subequations}
Observe the equation for $\dot{q}$ tells us how to recover the
velocities from the momenta $p_{n}$ (obeying the primary constraints
$\phi_{m}=0$) and the extra parameters $u^{m}$. These $u^{m}$ parameters
may be thought of as coordinates on the inverse image of a given
$p_{n}$.

\begin{prop}
If the primary constraints are independent, then the vectors
$\partial\phi_{m}/\partial p_{n}$ are also independent on the primary
constraint surface $\phi_{m}=0$.
\end{prop}
\begin{proof}
Take
\begin{equation}
\begin{split}
\frac{\partial}{\partial\dot{q}^{n}}\phi_{m}\bigl(q, p(q, \dot{q})\bigr)
&=\frac{\partial\phi_{m}}{\partial p_{n'}}\left(\frac{\partial p_{n'}}{\partial\dot{q}^{n}}\right)\\  
&=\frac{\partial\phi_{m}}{\partial p_{n'}}W_{nn'}
\end{split}
\end{equation}
but we see $\ker(\partial p_{n}/\partial\dot{q}^{n'})$ is spanned
by\dots well, it's enough to show $\partial\phi_{m}/\partial p_{n}$ are
independent if we use regularity conditions.
\end{proof}
\begin{cor}
No two different sets of $u$'s may express the same velocities.
\end{cor}

\M
We may write, in principle, the $u$'s as a function of positions and
velocities by solving
\begin{equation}
\dot{q}^{n} = \frac{\partial H}{\partial p_{n}}\bigl(q, p(q,\dot{q})\bigr)
+ u^{m}(q, \dot{q})%
\frac{\partial\phi_{m}}{\partial p_{n}}\bigl(q, p(q,\dot{q})\bigr).
\end{equation}

\M
Observe we may define the Legendre transform from $(q,\dot{q})$-space to
the $\phi_{m}(q,p)=0$ surface of $(q,p,u)$-space by means of
\begin{subequations}
\begin{align}
q^{n} &= q^{n}\\
p_{n} &= \frac{\partial L}{\partial\dot{q}^{n}}(q,\dot{q})\\
u^{m} &= u^{m}(q, \dot{q})
\end{align}
\end{subequations}
We can invert this Legendre transform
\begin{subequations}
\begin{align}
q^{n}         &= q^{n}\\
\dot{q}^{n}   &= \frac{\partial H}{\partial p_{n}} +
                 u^{m}\frac{\partial\phi_{m}}{\partial p_{n}}\\
\phi_{m}(q,p) &= 0
\end{align}
\end{subequations}
Thus we recover invertibility at the cost of adding more variables.

\N{Remark}
Observe all our considerations so far have been local. We assume the
results hold globally. This implies $H$ is not multi-valued.

The only modification we ought to make is to work with non-redundant
constraints. Then all our results carry over.

\subsection{Action Principle in Hamiltonian Form}
\N{Brief Review}
Just to recap our results, we have the original Lagrangian equations of
motion in Hamilton's form
\Mref{n:constrained:eom-in-hamiltons-form}. This is a direct consequence
of our analysis
\begin{equation}
  \delta\action=0%
  \implies\delta H
  -\left(\dot{q}^{n}\delta p_{n}
         -\frac{\partial L}{\partial q^{n}}\delta q^{n}\right)=0
\end{equation}
then applying theorem \ref{thm:variation-tangent-to-constraint-surface}
to restrict focus on the constraint surface.

We see
\begin{subequations}
\begin{equation}
\dot{q}^{n}=\frac{\partial H}{\partial p_{n}} +
u^{m}\frac{\partial\phi_{m}}{\partial p_{n}}
\end{equation}
and
\begin{equation}
\phi_{m}(q,p)=0
\end{equation}
together lets us recover
\begin{equation}
p_{n} = \frac{\partial L}{\partial\dot{q}^{n}}.
\end{equation}
\end{subequations}
When we insert this into the remaining equation
\begin{equation}
\dot{p}_{n} = -\frac{\partial H}{\partial q^{n}}-u^{m}\frac{\partial\phi_{m}}{\partial q^{n}},
\end{equation}
we recover the Lagrangian equations of motion precisely.

\N{Derived from Variational Principle}
We may derive Hamilton's equations of motion from the variational
principle
\begin{equation}
\delta\int^{t_{2}}_{t_{1}}\left(\dot{q}^{n}p_{n}-H-u^{m}\phi_{m}\right)\,\D t=0
\end{equation}
for arbitrary variations $\delta q^{n}$, $\delta p_{n}$, $\delta u^{m}$
subject to the restriction $\delta q^{n}(t_{1})=\delta q^{n}(t_{2})=0$.
Observe variation with respect to momenta produces
\begin{equation}
  \dot{q}^{n} = \frac{\partial H}{\partial p_{n}}
  +u^{m}\frac{\partial\phi_{m}}{\partial p_{n}}.
\end{equation}
Variation with respect to $\delta u^{m}$ gives
\begin{equation}
  \phi_{m}(q,p) = 0.
\end{equation}
Finally, variation with respect to the momenta gives
\begin{equation}
\dot{p}_{n} = -\frac{\partial H}{\partial q^{n}}
  -u^{m}\frac{\partial\phi_{m}}{\partial q^{n}}.
\end{equation}
Observe\marginpar{{\footnotesize\em $u^{m}$ as Lagrange multipliers}} then that the $u^{m}$ now serve as Lagrange multipliers
enforcing the constraints.

\N{Remark}
We may fix the endpoints of momentum in our variational analysis,
$\delta p_{n}(t_{1})=\delta p_{n}(t_{2})=0$ provided we replace
$\dot{q}^{n}p_{n}$ with $-q^{n}\dot{p}_{n}$. This comes in handy when
analyzing fermionic systems.

\M
We see the theory remains invariant under $H\to H+c^{m}\phi_{m}$, it
amounts to relabeling the Lagrange multipliers $u^{m}\to
u^{m}+c^{m}$. (To see this, just plug it into the action and perform the
change.)

\N{Solving the Constraints before Varying}
We may alternatively consider the good old fashioned action where we
first solve the constrsaints then perform the variation
\begin{equation}
\delta\int(p\dot{q}-H)\,\D t=0
\end{equation}
for independent vairations of position and momenta subject to
$\phi_{m}=0$ and $\delta\phi_{m}=0$. This is equivalent to the standard
Lagrange multipler approach we learned in basic calculus.

\emph{NB} the regularity conditions
\Mref{n:constrained-hamiltonian:regularity-cond}
play a critical role, otherwise solving the constraints before varying
would not be the same as varying then solving the constraints.

\N{Equations of Motion}\label{n:constrained-hamiltonian:poisson-bracket}
The equations of motion are here given by
\begin{equation}
  \dot{F} = \PoissonBracket{F}{H} + u^{m}\PoissonBracket{F}{\phi_{m}}
\end{equation}
where $F(q,p)$ is an arbitrary function of phase space variables, and
$\PoissonBracket{-}{-}$ is the standard Poisson bracket
\begin{equation}
  \PoissonBracket{A}{B} =
  \frac{\partial A}{\partial q^{n}}\frac{\partial B}{\partial p_{n}}
- \frac{\partial B}{\partial q^{n}}\frac{\partial A}{\partial p_{n}}.
\end{equation}

\subsection{Secondary Constraints}

\N{Consistency of Constraints}
A basic consistent requirement for the constraints is that they should
not change over time, i.e.,
\begin{equation}
  \dot{\phi}_{m}=0.
\end{equation}
We use the equations of motion \Mref{n:constrained-hamiltonian:poisson-bracket}
to write this as
\begin{equation}
\PoissonBracket{\phi_{m}}{H} + u^{m'}\PoissonBracket{\phi_{m}}{\phi_{m'}}=0.
\end{equation}
This may either (i) reduce to a relation independent of the $u$'s [hence
  involving only the $p$'s and $q$'s --- the case we will study here],
or (ii) impose a restriction on the $u$'s (which we will study in
subsection \ref{subsec:constrained-hamiltonian:restrictions-on-lagrange-multipliers}).

\begin{defn}
A \define{Secondary Constraint} is a relation of the $p$'s and $q$'s
(but not the $u$'s) from the consistency condition
\begin{equation}
\PoissonBracket{\phi_{m}}{H} + u^{m'}\PoissonBracket{\phi_{m}}{\phi_{m'}}=0.
\end{equation}
\end{defn}

\M
Observe the primary constraints came about because of the definition of
momenta
\begin{equation}
  p_{n} = \frac{\partial L}{\partial\dot{q}^{n}}.
\end{equation}
They \emph{did not} involve the equations of motion. The secondary
constraints emerged from the equations of motion of constraints.
%% (Some
%% authors then impose consistency on the secondary constraints, and if any
%% new constraints are found they're called ``tertiary''; consistency of
%% tertiary constraints are then imposed, and so on until we're done.)

\begin{puzzle}
Since the relation is independent of the $u$'s, does it suffice to say
the $\PoissonBracket{H}{\phi_{m}}$ gives us a secondary constraint?
\end{puzzle}

\N{Consistency of \emph{Secondary} Constraints}
We have now some secondary constraint, lets call it $X(q,p)$. We must
impose another consistency condition
\begin{equation}
\PoissonBracket{X}{H}+u^{m}\PoissonBracket{X}{\phi_{m}}=0.
\end{equation}
Again, this either (i) creates new constraints [aptly called ``Tertiary
  Constraints'' by some authors], or (ii) restricts the
$u$'s. (\emph{NB} some authors call tertiary constraints ``Secondary''
since they are derived from the equations of motion; Henneaux and
Teitelboim~\cite{Henneaux:1992ig} call every constraint derived from the
equations of motion ``Secondary'', for example.)

After this process is finished, we are left with a number of constraints
$\phi_{k}=0$ for $k=M+1$, \dots, $M+K$, where we have $K$ secondary
constraints. The reason for notation is because the difference between
primary and secondary constraints doesn't really matter for the
Hamiltonian formalism. (Does it matter in the Lagrangian formalism?) We
thus will denote all constraints $\phi_{j}=0$ for $j=1$, \dots,
$M+K=J$.

\N{Regularity}
We make the same regularity assumption for secondary constraints we made
for primary constraints \Mref{n:constrained-hamiltonian:regularity-cond}.
Namely: we assume $\phi_{j}=0$ defines a smooth submanifold but we also
take the constraint for $\phi_{j}$ to obey those very regularity
conditions. We will assume that $\PoissonBracket{\phi_{j}}{\phi_{j'}}$
is constant throughout the surface $\phi_{j}=0$ where the constraints
hold.

\subsection{Weak and Strong Equations}

\begin{defn}
We introduce \define{Weak Equality} for arbitrary phase space functions
$F$ and $G$ writing $F\weakEq G$ if and only if $F-G = c^{j}(q,p)\phi_{j}$
(where we sum $j=1$, \dots, $J$ over all constraints). It is an
equivalence relation.
\end{defn}

\begin{xca}
  Prove weak equality is indeed an equivalence relation.
\end{xca}

\M
We write $\phi_{j}\weakEq0$ to emphasize we restrict focus to when the
constraints vanish\dots as they are not guys that identically vanish
throughout phase space, they are merely numerically restricted to zero.

\begin{defn}
If a relation holds throughout phase space, then we use
\define{Strong Equality} --- the vanilla equality we use all the time,
writing $F=G$ and saying ``$F$ is `strongly equal' to $G$''.
\end{defn}

\subsection{Restrictions on the Lagrange Multipliers}\label{subsec:constrained-hamiltonian:restrictions-on-lagrange-multipliers}
\N{Consistency Condition As System of Linear Equations}
Suppose we have a complete set of constraints $\phi_{j}$, we now want to
consider the situation when
\begin{equation}
\PoissonBracket{\phi_{j}}{H} + u^{m}\PoissonBracket{\phi_{j}}{\phi_{m}}\weakEq0
\end{equation}
which restricts the $u$'s. We can look at this as a system of linear
equations, where we attempt to solve for the unknown $u$'s and
$\PoissonBracket{\phi_{j}}{H}$ is the constant vector.

\N{Generic Solution}
The general solution looks like
\begin{equation}
  u^{m} = U^{m} + V^{m}
\end{equation}
where $U^{m}$ is a particular solution, and $V^{m}$ is a solution to the
corresponding homogeneous system of equations. That is to say
\begin{equation}
V^{m}\PoissonBracket{\phi_{j}}{\Phi_{m}}\weakEq 0.
\end{equation}
We have $A$ such solutions, where $A$ is the rank of
$\PoissonBracket{\phi_{j}}{\phi_{m}}$ --- which we assume is constant on
the constraint surface.

So we have $V^{m}=v^{a}{V_{a}}^{m}$ for $a=1$, \dots, $A$ where $v^{a}$
are arbitrary functions of time, $q$'s, and $p$'s. They are completely
arbitrary! They represent the arbitrariness (or ``redundancies'') which
gauge systems possess. If $A=0$, the system possess no gauge symmetry.

The most general solution is then
\begin{equation}
u^{m}\weakEq U^{m} + v^{a}{V_{a}}^{m}.
\end{equation}

\subsection{Irreducible and Reducible Constraints}
\begin{defn}
If the constraints $\phi_{j}\weakEq0$ are not independent, we call them
\define{Reducible} (or \emph{``Redundant''}) and say we are in the
reducible case. If the constraints are independent, then they are
\define{Irreducible}.
\end{defn}

\M
Observe we may drop redundant constraints without any loss. So we may
suppose we're always (locally) in the irreducible case. But splitting
constraints into redundant ones and irreducible ones raises several
issues:
\begin{enumerate}
\item this separation may be awkward to do;
\item it may spoil some invariance of the system;
\item it may be impossible due to some topological obstruction.
\end{enumerate}

\M
Reducible constraints come about in, e.g., $p$-form gauge theories where
the $p$-form fields are part of the constraints. (See, e.g., Henneaux
and Teitelboim's paper~\cite{Henneaux:1986ht} on $p$-form electrodynamics.)

\M
We can always add redundant constraints to the theory without any
serious obstacle.

\subsection{Total Hamiltonian}
\begin{thm}
The equations of motion are
\begin{equation}
\dot{F}\weakEq\PoissonBracket{F}{H'+v^{a}\phi_{a}}
\end{equation}
where
\begin{equation}
  H' = H + U^{m}\phi_{m}
\end{equation}
and $\phi_{a} = {V_{a}}^{m}\phi_{m}$.
\end{thm}
(The $H'$ is called the \define{First-Class Hamiltonian}.)
\begin{proof}
We see explicitly
\begin{equation}
  \begin{split}
\PoissonBracket{F}{U^{m}\phi_{m}}&=U^{m}\PoissonBracket{F}{\phi_{m}}
  + \phi_{m}\PoissonBracket{F}{U^{m}}\\
&\weakEq U^{m}\PoissonBracket{F}{\phi_{m}}.
  \end{split}
\end{equation}
Similar reasoning shows
\begin{equation}
  \PoissonBracket{F}{{V_{a}}^{m}\phi_{m}}\weakEq {V_{a}}^{m}\PoissonBracket{F}{\phi_{m}}.
\end{equation}
Hence
\begin{equation}
  \PoissonBracket{F}{H'}\weakEq\PoissonBracket{F}{H}+U^{m}\PoissonBracket{F}{\phi_{m}}
\end{equation}
and thus
\begin{equation}
\PoissonBracket{F}{H'+v^{a}\phi_{a}}\weakEq\PoissonBracket{F}{H}+(U^{m}+v^{a}{V_{a}}^{m})\PoissonBracket{F}{\phi_{m}}  
\end{equation}
which is weakly the equations of motion.
\end{proof}
\begin{xca}
Show $\PoissonBracket{H'}{\phi_{j}}\weakEq0$.
\end{xca}
\begin{defn}
  The \define{Total Hamiltonian} is
  \begin{equation}
    \totalHamiltonian := H' + v^{a}\phi_{a} = H + u^{m}\phi_{m}
  \end{equation}
  i.e., the sum of the canonical Hamiltonian $H$ and the sum of primary
  constraints multiplied by the $u$'s, i.e., $u^{m}\phi_{m}$.
\end{defn}
\N*{Remark} We have several observations
\begin{enumerate}
\item The equations of motion are
  $\dot{F}\weakEq\PoissonBracket{F}{\totalHamiltonian}$.
\item The $v^{a}$ functions reflect the arbitrariness from the gauge
  symmetries in our system.
\end{enumerate}
