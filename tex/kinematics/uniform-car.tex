

\begin{ex}[{Irodov~\cite[1.3]{irodovProblems}}]
A car starts moving rectilinearly, first with acceleration
$w=\SI[per-mode=symbol]{5.0}{\meter\per\second\squared}$ (the initial
velocity is equal to zero), then uniformly, and finally, decelerating at
the same rate $w$, comes to a stop. The total time of motion equals
$\tau=\SI{25}{\sec}$. The average during that time is equal to $\langle
v\rangle=\SI[per-mode=symbol]{72}{\kilo\meter\per\hour}$. How long does
the car move uniformly?
\end{ex}
\begin{soln}
\IDENTIFY
We have $w=\SI{5.0}{\meter\per\second\squared}$
describe the acceleration and deceleration of the car. Let $\Delta t_{0}$
be the time interval when the car is accelerating,
$\Delta t_{1}$ be the interval when there is no acceleration (i.e., the
car experiences ``uniform motion''), and $\Delta t_{2}$ be the time
interval when the car is decelerating. These intervals are unknown but
positive, and satisfy $\tau = \Delta t_{0} + \Delta t_{1} + \Delta t_{2}$
for $\tau=\SI{25}{\sec}$.
Symmetry suggests $\Delta t_{2}=\Delta t_{0}$. Last, let $v$ be the
(unknown) velocity when the car is moving uniformly.

\SETUP
We have
\begin{equation}
\ell = \tau \langle v\rangle
\end{equation}
Uniform acceleration gives the velocity
\begin{equation}
v = w\Delta t_{0}
\end{equation}
The distance traveled while accelerating uniformly is
\begin{equation}
d_{0} = \frac{1}{2}w(\Delta t_{0})^{2}.
\end{equation}
Conversely, while decelerating, the distance traveled is $d_{2}=d_{0}$.
The distance traveled under uniform velocity is
\begin{equation}
d_{1} = v\Delta t_{1}.
\end{equation}
The average velocity is then by definition
\begin{equation}\label{eq:ir1.3:def-avg-velocity}
\langle v\rangle = \frac{2d_{0} + d_{1}}{2\Delta t_{0} + \Delta t_{1}}
\end{equation}

\EXECUTE
We expand the definitions of $d_{0}$ and $d_{1}$ in Eq \eqref{eq:ir1.3:def-avg-velocity}
to find
\begin{subequations}
\begin{align}
\frac{2d_{0} + d_{1}}{2\Delta t_{0} + \Delta t_{1}}
&= \frac{w (\Delta t_{0})^2 + w\Delta t_{0}\Delta t_{1}}{2\Delta t_{0} + \Delta t_{1}}\\
&= \frac{(w \Delta t_{0}) (\Delta t_{0} + \Delta t_{1})}{2\Delta t_{0} + \Delta t_{1}}\\
&= (w \Delta t_{0})\left(1 - \frac{\Delta t_{0}}{2\Delta t_{0} + \Delta t_{1}}\right)\\
&= (w \Delta t_{0})\left(1 - \frac{\Delta t_{0}}{\tau}\right).
\end{align}
\end{subequations}
This gives us a quadratic equation in $\Delta t_{0}$
\begin{subequations}
\begin{equation}
\langle v\rangle = -(w/\tau)(\Delta t_{0})^{2} + w\Delta t_{0}
\end{equation}
Multiplying through by $-\tau/w$
\begin{equation}
-\frac{\tau\langle v\rangle}{w} = (\Delta t_{0})^{2} - \tau\Delta t_{0}
\end{equation}
then completing the square on the right hand side
\begin{equation}
-\frac{\tau\langle v\rangle}{w} = (\Delta t_{0} - \tau/2)^{2} - \tau^{2}/4.
\end{equation}
Rearranging terms yields
\begin{equation}
\frac{\tau^{2}}{4}-\frac{\tau\langle v\rangle}{w} = (\Delta t_{0} - \tau/2)^{2}
\end{equation}
and hence taking the squareroot of both sides
\begin{equation}
\Delta t_{0} - \frac{\tau}{2}=\pm\sqrt{\frac{\tau^{2}}{4}-\frac{\tau\langle v\rangle}{w}}.
\end{equation}
The left hand side is non-positive (since $\tau/2 = \Delta t_{1}/2 + \Delta t_{0} > \Delta t_{0}$),
which means we must take the negative value of the squareroot
\begin{equation}
\Delta t_{0} - \frac{\tau}{2}=-\sqrt{\frac{\tau^{2}}{4}-\frac{\tau\langle v\rangle}{w}}.
\end{equation}
Hence (multiplying through by $-2$)
\begin{equation}
\tau - 2\Delta t_{0} = 2\sqrt{\frac{\tau^{2}}{4}-\frac{\tau\langle v\rangle}{w}}.
\end{equation}
\end{subequations}
This is the solution for $\Delta t_{1}$
\begin{equation}
\begin{split}
\Delta t_{1} &= 2\sqrt{\frac{\tau^{2}}{4}-\frac{\tau\langle v\rangle}{w}}\\
&= \tau\sqrt{1-\frac{4\langle v\rangle}{w\tau}}.
\end{split}
\end{equation}

\EVALUATE
We see when
\begin{equation}
\frac{\tau^{2}}{4}=\frac{\tau\langle v\rangle}{w}
\end{equation}
the time the car is moving uniformly is $\Delta t_{1}=0$.
Also, as $w\to\infty$ or $\langle v\rangle\to0$, we have $\Delta t_{1}\to 0$. We also have a reality condition
\begin{equation}
\frac{w\tau}{4}\geq\langle v\rangle
\end{equation}
otherwise we end up with $\Delta t_{1}\in\I\RR$ which is unphysical.
\end{soln}