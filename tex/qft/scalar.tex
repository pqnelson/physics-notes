\N*{Hokey Motivation}
We have quantized the Schrodinger equation to get a non-relativistic
quantum field theory, i.e., one invariant under Galilean transformations
\emph{but not} Lorentz transformations.

We will consider a relativistic Schrodinger equation, then we will
second-quantize it. How to get a relativistic Schrodinger equation? One
possibility is to guess. The other is to ``first quantize'' a classical
relativistic particle traveling along a geodesic. In either case, we
get the same result: the Klein-Gordon equation. It describes a scalar
field.

Since the Lorentz group may be ``built'' from $\SU{2}$ (the ``spin''
group), we may conclude safely that we have a spin-$0$ field. Later we
will examine spin-$1/2$ and spin-$1$ fields.

\section{Klein-Gordon Equation}
\M
The Schrodinger equation for a free particle may be heuristically
obtained via the quantization of the kinetic energy
\begin{equation}
E = \frac{p^{2}}{2m}
\end{equation}
via the substitution $E\to\qop{E}=\I\partial_{t}$,
$p_{j}\to\qop{p}_{j}=-\I\partial_{j}$. This energy-momentum relation
above is non-relativistic. So, we consider the relativistic version of
it
\begin{equation}\label{eq:klein-gordon:mass-shell-relation}
E^{2} = p^{2} + m^{2}.
\end{equation}
Quantizing this system gives us
\begin{equation}\label{eq:klein-gordon:kg-eqn}
(\partial_{t}^{2}-\nabla^{2})\varphi(x,t) + m^{2}\varphi(x,t)=0.
\end{equation}
We may derive this from the action
\begin{equation}
\action[\varphi]=\frac{1}{2}\int\bigl(\partial_{\mu}\varphi\partial^{\mu}\varphi-m^{2}\varphi^{2}\bigr)\D^{4}x
\end{equation}
where $x=(t,\vec{x})$, $x^{0}=t$. Observe $\action[\varphi]$ transforms
as a Lorentz scalar if $\varphi$ transforms as one. Equivalently, the
Klein-Gordon equation is Lorentz covariant\footnote{Regarding
  terminology, we shall call something ``invariant'' under some change
  of coordinates if it remains ``exactly the same'', i.e., we need not
  multiply by any number of Jacobians or their transpose. Lorentz
  covariance here implies we need some suitable action of the Lorentz group.}
if $\varphi$ acts as a scalar.

\begin{ddanger}
\textbf{CAUTION:}
I will randomly rewrite $x=(t,\vec{x})$ as $x=(\vec{x},t)$ with the
understanding that $t$ \emph{always} refers to the time part. I do this
because I'm inconsistent.
\end{ddanger}

\N{Problem: Negative Energies}
We see we could get negative energies, since the relation
\eqref{eq:klein-gordon:mass-shell-relation} is invariant under
$E\to-E$. We see we have the plane wave solution
$\exp[\I(\vec{k}\cdot\vec{x}+\omega_{\vec{k}}{t})]$, it is a solution with energy
\begin{equation}
E = -\omega_{\vec{k}} = -(\vec{k}^{2}+m^{2})^{1/2} < 0.
\end{equation}
What does this mean physically? Worse, energy is unbounded from
below. We may, in principle, extract an infinite amount of energy from a
one-particle system\dots which is clearly unphysical.

If we try to avoid this negative energy problem by taking the positive
square-root
\begin{equation}
E=+\sqrt{\vec{p}^{2}+m^{2}},
\end{equation}
we have to series expand it\dots and we get nonlocal operator
expressions!

\N{Problem: Probabilities}
Lets try to interpret $\varphi$ as a wave function, i.e., a probability
amplitude. It must be a conserved quantity, or give rise to one---just
as $\varphi^{*}\varphi$ did for the Schrodinger equation
\Mref{par:schrodinger-conserved-current}.

The Klein-Gordon equation \emph{does} have a conserved quantity, which
is (more or less)
\begin{equation}
j_{\ell} = \frac{1}{2\I m}(\varphi^{*}\partial_{\ell}\varphi-(\partial_{\ell}\varphi^{*})\varphi).
\end{equation}
The time component, however, is
\begin{equation}
j_{0} = \rho = \frac{\I}{2m}(\varphi^{*}\partial_{t}\varphi-(\partial_{t}\varphi^{*})\varphi).
\end{equation}
The probability density is not positive-definite. For example, the
plane wave has its probability density be
\begin{equation}
\rho=\frac{-\omega_{\vec{k}}}{m}=\frac{-1}{m}\sqrt{\vec{k}^{2}+m^{2}}<0.
\end{equation}
We thus get negative probabilities!

But these problems arise because we are trying to interpret the
Klein-Gordon equation as a 1-particle state. Lorentz invariance prevents
us from doing so.

\section{Free Scalar Field} % page 17

\M
We want to canonically quantize the field $\varphi(x)$. We want to turn
$\varphi$ into an operator (or ``operator-valued density'', if you're a
mathematician). Given the classical action
\begin{equation}
\action_{KG}[\varphi]=\frac{1}{2}\int\left(\partial_{\mu}\varphi(x)%
\partial^{\mu}\varphi(x)-m^{2}\varphi(x)^{2}\right)\D^{4}x
\end{equation}
we can find the canonically conjugate momenta
\begin{equation}
\pi(x)=\frac{\partial\mathcal{L}}{\partial(\partial_{t}\varphi(x))}=\dot{\varphi}(x).
\end{equation}
Now we can find the Hamiltonian (density) via the Legendre transform
\begin{subequations}
\begin{align}
\mathcal{H} 
&=\pi(x)\dot\varphi(x) - \mathcal{L}\\
&=\pi(x)\dot{\varphi}(x)
   -\frac{1}{2}\partial_{\mu}\varphi\partial^{\mu}\varphi
   +\frac{1}{2}m^{2}\varphi(x)^{2}\\
&=\pi(x)\dot{\varphi}(x)
  - \frac{1}{2}\bigl[\partial_{t}\varphi\partial^{t}\varphi-|\nabla\varphi|^{2}\bigr]
  + \frac{1}{2}m^{2}\varphi(x)^{2}\\
&=\pi(x)^{2}
  - \frac{1}{2}\bigl[\pi(x)^{2}-|\nabla\varphi|^{2}\bigr]
  + \frac{1}{2}m^{2}\varphi(x)^{2}\\
&=\frac{1}{2}\pi(x)^{2}
  +\frac{1}{2}|\nabla\varphi(x)|^{2}
  +\frac{m^{2}}{2}\varphi(x)^{2}.
\end{align}
\end{subequations}
The Hamiltonian is thus
\begin{equation}
\begin{split}
H(t) &\eqDef\int\mathcal{H}(\vec{x},t)\,\D^{3}x\\
&=\frac{1}{2}\int\left(\pi(\vec{x},t)^{2}+|\nabla\varphi(\vec{x},t)|^{2}+m^{2}\varphi(\vec{x},t)^{2}\right)\D^{3}x.
\end{split}
\end{equation}
We also have the (quantum) commutator relations
\begin{equation}
\begin{split}
\commutator{\varphi(\vec{x},t)}{\pi(\vec{y},t)}=\I\delta^{(3)}(\vec{x}-\vec{y})\\
\commutator{\varphi(\vec{x},t)}{\varphi(\vec{y},t)}=
\commutator{\pi(\vec{x},t)}{\pi(\vec{y},t)}=0
\end{split}
\end{equation}


\begin{xca}
(a) We will derive the definition of the conjugate momenta using the
  quantum Poisson bracket relation. So given
\begin{equation}
\dot{\varphi}=\I\commutator{H}{\pi}
\end{equation}
we will show this reproduces the definition for the conjugate momenta.

(b) We will derive the equations of motion, using the quantum Poisson
bracket relations. Given
\begin{equation}
\dot{\pi}=\I\commutator{H}{\varphi}
\end{equation}
show this reproduces the equations of motion.
\end{xca}
\begin{soln}
(a) We see
\begin{equation}
\begin{split}
2\I\commutator{H}{\varphi(\vec{y},t)}
&=\int\commutator{\pi(\vec{x},t)^{2}}{\varphi(\vec{y},t)}\D^{3}x
+\underbrace{\int\bigO{\commutator{\varphi(\vec{x},t)}{\varphi(\vec{y},t)}}\D^{3}x}_{=0}\\
&=\int\commutator{\pi(\vec{x},t)^{2}}{\varphi(\vec{y},t)}\D^{3}x
\end{split}
\end{equation}
Hence we find
\begin{subequations}
\begin{align}
\int\commutator{\pi(\vec{x},t)^{2}}{\varphi(\vec{y},t)}\D^{3}x
&=\int\pi(\vec{x},t)\commutator{\pi(\vec{x},t)}{\varphi(\vec{y},t)}\D^{3}x\nonumber\\
&\quad+\int\commutator{\pi(\vec{x},t)}{\varphi(\vec{y},t)}\pi(\vec{x},t)\D^{3}x\\
&=\int\pi(\vec{x},t)\I\delta^{(3)}(\vec{x}-\vec{y})\D^{3}x\nonumber\\
&\quad+\int\I\delta^{(3)}(\vec{x}-\vec{y})\pi(\vec{x},t)\D^{3}x\\
&=\I 2\pi(\vec{y},t)
\end{align}
\end{subequations}
which implies
\begin{equation}
2\I\commutator{H}{\varphi(\vec{y},t)}=\I 2\pi(\vec{y},t).
\end{equation}
We see the right hand side of this equation is, by Hamilton's equations
in Poisson bracket form, necessarily
$\partial_{t}\varphi(\vec{y},t)$. This implies
\begin{equation}
\partial_{t}\varphi(\vec{y},t)=\pi(\vec{y},t)
\end{equation}
as desired. \emph{So we have just derived the definition of the conjugate momenta using nothing more than commutators and Hamilton's equations in Poisson bracket form!}

(b) We see by direct computation
\begin{equation}
\begin{split}
2\I\commutator{H}{\varphi(\vec{y},t)}&=
 \I\int\commutator{\pi(\vec{x},t)^{2}}{\pi(\vec{y},t)}\D^{3}x\\
&\quad+\I\int\commutator{\partial_{j}\varphi(\vec{x},t)\partial^{j}\varphi(\vec{x},t)}{\pi(\vec{y},t)}\D^{3}x\\
&\quad+\I m^{2}\int\commutator{\varphi(\vec{x},t)^{2}}{\pi(\vec{y},t)}\D^{3}x.
\end{split}
\end{equation}
Consider each term, we find immediately
\begin{equation}
I_{1}\eqDef\int\commutator{\pi(\vec{x},t)^{2}}{\pi(\vec{y},t)}\D^{3}x=0
\end{equation}
thanks to the commutator relations. Next we find
\begin{equation}
\commutator{\partial_{k}\varphi(\vec{x},t)}{\pi(\vec{y},t)}=\partial_{k}\delta^{(3)}(\vec{x}-\vec{y}).
\end{equation}
This simplifies the second term
\begin{subequations}
\begin{align}
I_{2}&\eqDef\int\commutator{\partial_{j}\varphi(\vec{x},t)\partial^{j}\varphi(\vec{x},t)}{\pi(\vec{y},t)}\D^{3}x\\
&=\int\partial_{j}\varphi(\vec{x},t)\commutator{\partial^{j}\varphi(\vec{x},t)}{\pi(\vec{y},t)}+\commutator{\partial_{j}\varphi(\vec{x},t)}{\pi(\vec{y},t)}\partial^{j}\varphi(\vec{x},t)\D^{3}x\\
&=\I\int\partial_{j}\varphi(\vec{x},t)\partial^{j}\delta^{(3)}(\vec{x}-\vec{y})+\partial_{j}\delta^{(3)}(\vec{x}-\vec{y})\partial^{j}\varphi(\vec{x},t)\D^{3}x\\
\intertext{then integrating by parts isolates the delta functions}
&=-\I\int\nabla^{2}\varphi(\vec{x},t)\delta^{(3)}(\vec{x}-\vec{y})+\delta^{(3)}(\vec{x}-\vec{y})\nabla^{2}\varphi(\vec{x},t)\D^{3}x\\
&=-2\I\nabla^{2}\varphi(\vec{y},t).
\end{align}
\end{subequations}
The last term is quite simply
\begin{subequations}
\begin{align}
I_{3} &\eqDef\int\commutator{\varphi(\vec{x},t)^{2}}{\pi(\vec{y},t)}\D^{3}x\\
&=\int\varphi(\vec{x},t)\commutator{\varphi(\vec{x},t)}{\pi(\vec{y},t)}+\commutator{\varphi(\vec{x},t)}{\pi(\vec{y},t)}\varphi(\vec{x},t)\D^{3}x\\
&=\int\varphi(\vec{x},t)\I\delta^{(3)}(\vec{x}-\vec{y})+\I\delta^{(3)}(\vec{x}-\vec{y})\varphi(\vec{x},t)\,\D^{3}x\\
&=2\I\varphi(\vec{y},t).
\end{align}
\end{subequations}
Thus we find
\begin{equation}
\begin{split}
2\I\commutator{H}{\pi(\vec{y},t)}&=\I I_{1}+\I I_{2} + \I m^{2} I_{3}\\
&=0 + 2\nabla^{2}\varphi(\vec{y},t) - 2m^{2}\varphi(\vec{y},t).
\end{split}
\end{equation}
But we see the right hand side is --- by definition ---
$2\partial_{t}\pi(\vec{y}, t)$. So, setting equals to equals we find
\begin{equation}
2\partial_{t}\pi(\vec{y}, t)=2\nabla^{2}\varphi(\vec{y},t) - 2m^{2}\varphi(\vec{y},t).
\end{equation}
Look: this is precisely Eq \eqref{eq:klein-gordon:kg-eqn} using $\pi$
instead of $\partial_{t}\varphi$. So, we have derived the equations of
motion from the commutator relations and the Hamiltonian.
\end{soln}

\N{Plane Wave Expansions}\label{par:scalar:plane-wave-expansions}
We now expand $\varphi(\vec{x},t)$ in terms of the classical solutions.
\begin{equation}\label{eq:scalar:plane-wave-expansion-eqn}
\begin{split}
\varphi(\vec{x},t)
\;\,&\mbox{``$=$''}\;\sum_{\vec{k}}a(\vec{k})\varphi_{\vec{k}}^{(+)}(\vec{x},t)
                          + b(\vec{k})\varphi_{\vec{k}}^{(-)}(\vec{x},t)\\
&=\int(a(k)\E^{-\I k_{\mu}x^{\mu}}+b(k)\E^{\I k_{\mu}x^{\mu}})\frac{1}{2\omega_{\vec{k}}}\frac{\D^{3}k}{(2\pi)^{3}}
\end{split}
\end{equation}
where $k^{0}=\omega_{\vec{k}}=(\vec{k}^{2}+m^{2})^{1/2}$,
$\varphi^{(+)}(x)$ is the classical solution with positive energy, and
$\varphi^{(-)}(x)$ is the classical solution with negative energy. When
$\varphi$ becomes an operator, we expect $a(k)$ and $b(k)$ to be
operators.

We use $\varphi^{(+)}$ and $\varphi^{(-)}$ to guarantee $\varphi(x)$ is
a solution of the Klein-Gordon equation. This depends critically on the
Klein-Gordon equation being a linear differential equation.

\N{Volume-Element}
The normalization factor $1/2\omega_{\vec{k}}$ may be explained as the
volume element preserving Lorentz invariance. We rewrite the volume as
\begin{equation}\label{eq:scalar:volume-element-identity}
\int2\pi\delta^{(4)}(k^{2}-m^{2})\stepFn(k^{0})\frac{d^{4}k}{(2\pi)^{4}}
=\int\frac{1}{2\omega_{\vec{k}}}\frac{\D^{3}k}{(2\pi)^{3}}
\end{equation}
which is Lorentz invariant, where
\begin{equation}
\stepFn(x) = \begin{cases}0 & x<0\\
1 & x\geq0
\end{cases}
\end{equation}
is the Heaviside step function\index{$\stepFn(-)$}\index{Heaviside Step Function}\index{Step Function}.
We see Eq \eqref{eq:scalar:volume-element-identity} follows from the
identity
\begin{equation}
\delta\bigl(f(x)\bigr) = \sum_{j}\frac{1}{|f'(x_{j})|}\delta(x-x_{j})
\end{equation}
where $x_{j}$ is a simple zero of $f(x)$.

Observe
\begin{equation}
k^{2}-m^{2}=(k_{0})^{2}-{\omega_{\vec{k}}}^{2}
\end{equation}
so we may factorize the right hand side as
\begin{equation}
(k_{0})^{2}-{\omega_{\vec{k}}}^{2}=(k_{0}+\omega_{\vec{k}})(k_{0}-\omega_{\vec{k}}).
\end{equation}
Invoking the Delta function identity, we find
\begin{subequations}
\begin{align}
\delta^{(4)}(k^{2}-m^{2})
&=\delta({k_{0}}^{2}-{\omega_{\vec{k}}}^{2})\\
&=\frac{1}{2\omega_{\vec{k}}}\delta(k_{0}-\omega_{\vec{k}})
+\frac{1}{(-2\omega_{\vec{k}})}\delta(k_{0}+\omega_{\vec{k}})\\
&=\frac{1}{2\omega_{\vec{k}}}\bigl(\delta(k_{0}-\omega_{\vec{k}})-\delta(k_{0}+\omega_{\vec{k}})\bigr).\label{eq:scalar:delta-fn-manipulation:vol-elt}
\end{align}
\end{subequations}
the step function factor $\stepFn(k^{0})$ then picks out the
$\delta(k_{0}-\omega_{\vec{k}})$ term.

Hence plugging in Eq \eqref{eq:scalar:delta-fn-manipulation:vol-elt} into \eqref{eq:scalar:volume-element-identity}
gives us
\begin{equation}
\int2\pi\delta^{(4)}(k^{2}-m^{2})\stepFn(k^{0})\frac{d^{4}k}{(2\pi)^{4}}
=\int2\pi\frac{\delta(k_{0}-\omega_{\vec{k}})}{2\omega_{\vec{k}}}\stepFn(k^{0})\frac{d^{4}k}{(2\pi)^{4}}.
\end{equation}
Performing the integration over $k_{0}$
\begin{equation}
\int2\pi\frac{\delta(k_{0}-\omega_{\vec{k}})}{2\omega_{\vec{k}}}\stepFn(k^{0})\frac{d^{4}k}{(2\pi)^{4}}
=\int\frac{1}{2\omega_{\vec{k}}}\frac{\D^{3}k}{(2\pi)^{3}}
\end{equation}
which precisely demonstrates the equality in \eqref{eq:scalar:volume-element-identity}.

\begin{xca}
Recall \Mref{par:scalar:plane-wave-expansions} the plane-wave
expansion for $\varphi$. What is $\pi(\vec{x},t)$ in terms of $a(k)$ and $b(k)$?
\end{xca}
\begin{soln}
We see that
\begin{equation}
\pi(\vec{x},t)=\partial_{t}\varphi(\vec{x},t).
\end{equation}
So we find, using Eq \eqref{eq:scalar:plane-wave-expansion-eqn},
\begin{subequations}
\begin{align}
\pi(\vec{x},t)
&=\partial_{t}\varphi(\vec{x},t)\nonumber\\
&=\partial_{t}\int(a(k)\E^{-\I k_{\mu}x^{\mu}}+b(k)\E^{\I k_{\mu}x^{\mu}})\frac{1}{2\omega_{\vec{k}}}\frac{\D^{3}k}{(2\pi)^{3}}\\
&=\int(-\I\omega_{\vec{k}}a(k)\E^{-\I k_{\mu}x^{\mu}}+\I\omega_{\vec{k}}b(k)\E^{\I k_{\mu}x^{\mu}})\frac{1}{2\omega_{\vec{k}}}\frac{\D^{3}k}{(2\pi)^{3}}.
\end{align}
\end{subequations}
Thus we have found the expansion for the conjugate momenta.
\end{soln}

\N{Determining $a(k)$}
Lets now try to consider how we can express $a(k)$ in terms of
$\varphi(\vec{x},t)$ and $\pi(\vec{x},t)$. We expect
\begin{equation}
I_{1}(k')\eqDef\int\left(\omega_{\vec{k}'}\varphi(\vec{x},t)+\I\pi(\vec{x},t)\right)\E^{-\I\vec{k}'\cdot\vec{x}}\,\D^{3}x\looksLike a(k').
\end{equation}
Lets evaluate the left hand side, and see what we get. Plugging in the
expansions for $\varphi$ and $\pi$ gives us
\begin{equation}
\begin{split}
I_{1}(k')&=\iint\left[
(\omega_{\vec{k}'}+\omega_{\vec{k}})a(k)\E^{-\I k_{\mu}x^{\mu}}
+(\omega_{\vec{k}'}-\omega_{\vec{k}})b(k)\E^{\I k_{\mu}x^{\mu}}
\right]\frac{\E^{-\I\vec{k}'\cdot\vec{x}}}{2\omega_{\vec{k}}}\frac{\D^{3}x\,\D^{3}k}{(2\pi)^{3}}\\
&=\iint\left[
(\omega_{\vec{k}'}+\omega_{\vec{k}})a(k)\E^{-\I \omega_{\vec{k}}t +\I(\vec{k}-\vec{k}')\cdot\vec{x}}
+(\omega_{\vec{k}'}-\omega_{\vec{k}})b(k)\E^{\I \omega_{\vec{k}}t - \I(\vec{k}+\vec{k}')\cdot\vec{x}}
\right]\frac{1}{2\omega_{\vec{k}}}\frac{\D^{3}x\,\D^{3}k}{(2\pi)^{3}}
\end{split}
\end{equation}
Performing the integral over $x$ turns the exponential factors
$\exp(\I(\vec{k}\pm\vec{k}')\cdot\vec{x})$ into delta functions
$\delta^{(3)}(\vec{k}\pm\vec{k}')$. So we find
\begin{equation}
I_{1}(k')=\int\left[
\delta(\vec{k}-\vec{k}')(\omega_{\vec{k}}+\omega_{\vec{k}'})a(k)\E^{-\I\omega_{\vec{k}}t}
+\delta(\vec{k}+\vec{k}')(\omega_{\vec{k}'}-\omega_{\vec{k}})b(k)\E^{\I \omega_{\vec{k}}t}
\right]\frac{\D^{3}k}{2\omega_{\vec{k}}}
\end{equation}
The delta functions simplify things considerably, we get
\begin{equation}
\begin{split}
I_{1}(k')&=\left[
(2\omega_{\vec{k}'})a(k')\E^{-\I\omega_{\vec{k}'}t}
+(0)b(k')\E^{\I \omega_{\vec{k}'}t}
\right]\frac{1}{2\omega_{\vec{k}'}}\\
&=a(k')\exp(-\I\omega_{\vec{k}'}t).
\end{split}
\end{equation}
So we find
\begin{equation}\label{eq:scalar:ladder-operator-in-terms-of-field-vars}
\begin{split}
a(k) &= \E^{\I\omega_{\vec{k}}t}\int\left(\omega_{\vec{k}}\varphi(\vec{x},t)+\I\pi(\vec{x},t)\right)\E^{-\I\vec{k}\cdot\vec{x}}\,\D^{3}x\\
&=\int\left(\omega_{\vec{k}}\varphi(\vec{x},t)+\I\pi(\vec{x},t)\right)\E^{\I k_{\mu}x^{\mu}}\,\D^{3}x
\end{split}
\end{equation}

\N{Determining $b(k)$}
Great, so lets try to do the same thing with $b(k)$. We expect 
\begin{equation}
I_{2}(k')\eqDef\int\left(\omega_{\vec{k}'}\varphi(\vec{x},t)-\I\pi(\vec{x},t)\right)\E^{\I\vec{k}'\cdot\vec{x}}\,\D^{3}x\looksLike b(k').
\end{equation}
Performing the exact same steps as before, we find
\begin{equation}
I_{2}(k') = b(k')\exp(\I\omega_{\vec{k}'}t)
\end{equation}
hence
\begin{equation}
\begin{split}
b(k') &= \E^{-\I\omega_{\vec{k}'}t}\int\left(\omega_{\vec{k}'}\varphi(\vec{x},t)-\I\pi(\vec{x},t)\right)\E^{\I\vec{k}'\cdot\vec{x}}\,\D^{3}x\\
&=
\int\left(\omega_{\vec{k}'}\varphi(\vec{x},t)-\I\pi(\vec{x},t)\right)\E^{-\I k'_{\mu}x^{\mu}}\,\D^{3}x
\end{split}
\end{equation}
Observe, though, this resembles Eq
\eqref{eq:scalar:ladder-operator-in-terms-of-field-vars}\dots which is
unsurprising. But it appears that $b(k)$ is just the complex conjugate
of $a(k)$, i.e.,
\begin{equation}
b(k) = \bigl(a(k)\bigr)^{*}.
\end{equation}
So we\marginpar{\footnotesize\em CHANGE OF NOTATION $b(k)\to a^{\dagger}(k)$}
change our notation to reflect this relationship between $b(k)$ and
$a(k)$, using
\begin{equation}
b(k)\to a^{\dagger}(k).
\end{equation}
We will not use $b(k)$ henceforth when discussing the Scalar field. So,
to reiterate, we have
\begin{equation}\label{eq:scalar:adjoint-ladder-operator-in-terms-of-field-vars}
a^{\dagger}(k') = \int\left(\omega_{\vec{k}'}\varphi(\vec{y},t)-\I\pi(\vec{y},t)\right)\E^{-\I k'_{\mu}y^{\mu}}\,\D^{3}y
\end{equation}

\N{Differences From Harmonic Oscillator}
We have obtained ladder operators for our field. But we
should note, the simple harmonic oscillator has
\begin{equation}
a_{\text{SHO}}\looksLike \sqrt{\omega}x + \frac{\I}{\sqrt{\omega}}p
\end{equation}
whereas for the scalar field
\begin{equation}
a_{\text{scalar}}\looksLike \omega\varphi + \I\pi.
\end{equation}
This is \emph{not} a typo or error, this is just a difference
between the scalar field and the simple harmonic oscillator.

\begin{theorem}[Commutation Relations]\label{thm:scalar:ladder-op-commutation-relations}
We see the coefficient operators obey commutation relations
\begin{enumerate}[label=(\arabic*)]
\item $\commutator{a(k)}{a^{\dagger}(k')}=2\omega_{\vec{k}}\cdot(2\pi)^{3}\delta^{(3)}(\vec{k}-\vec{k}').$
\item $\commutator{a(k)}{a(k')}=0$
\item $\commutator{a^{\dagger}(k)}{a^{\dagger}(k')}=0$
\end{enumerate}
\end{theorem}
\begin{proof} These problems are trivial thanks to Eqs
  \eqref{eq:scalar:ladder-operator-in-terms-of-field-vars} and 
  \eqref{eq:scalar:adjoint-ladder-operator-in-terms-of-field-vars}.

% a(k) = \int\left(\omega_{\vec{k}}\varphi(\vec{x},t)+\I\pi(\vec{x},t)\right)\E^{\I k_{\mu}x^{\mu}}\,\D^{3}x
% b(k) =  \int\left(\omega_{\vec{k}'}\varphi(\vec{y},t)-\I\pi(\vec{y},t)\right)\E^{-\I k'_{\mu}y^{\mu}}\,\D^{3}y
(1) We find, after plugging in the integral expressions, we consider the
  commutator of the integrands
\begin{equation*}
\commutator{a(k)}{a^{\dagger}(k')}
=\int\commutator{\omega_{\vec{k}}\varphi(\vec{x},t)+\I\pi(\vec{x},t)}{\omega_{\vec{k}'}\varphi(\vec{y},t)-\I\pi(\vec{y},t)}\E^{\I (k_{\mu}x^{\mu}-k'_{\mu}y^{\mu})}\D^{3}y\,\D^{3}x
\end{equation*}
Linearity ``simplifies'' this
\begin{equation}\label{eq:proof:ladder-operator-commutation:expansion}
\begin{split}
\commutator{a(k)}{a^{\dagger}(k')}
=&\int\commutator{\omega_{\vec{k}}\varphi(\vec{x},t)}{\omega_{\vec{k}'}\varphi(\vec{y},t)}\E^{\I (k_{\mu}x^{\mu}-k'_{\mu}y^{\mu})}\D^{3}y\,\D^{3}x\\
&-\int\commutator{\omega_{\vec{k}}\varphi(\vec{x},t)}{\I\pi(\vec{y},t)}\E^{\I (k_{\mu}x^{\mu}-k'_{\mu}y^{\mu})}\D^{3}y\,\D^{3}x\\
&+\int\commutator{\I\pi(\vec{x},t)}{\omega_{\vec{k}'}\varphi(\vec{y},t)}\E^{\I (k_{\mu}x^{\mu}-k'_{\mu}y^{\mu})}\D^{3}y\,\D^{3}x\\
&-\int\commutator{\I\pi(\vec{x},t)}{\I\pi(\vec{y},t)}\E^{\I (k_{\mu}x^{\mu}-k'_{\mu}y^{\mu})}\D^{3}y\,\D^{3}x
\end{split}
\end{equation}
Our usual commutation relations keeps only the second and third terms,
which become (factoring out the common $\I$ factors from the momenta)
\begin{align}
\commutator{a(k)}{a^{\dagger}(k')}
&=-\I\int\left(\commutator{\omega_{\vec{k}}\varphi(\vec{x},t)}{\pi(\vec{y},t)}
+\commutator{\omega_{\vec{k}'}\varphi(\vec{y},t)}{\pi(\vec{x},t)}\right)\E^{\I (k_{\mu}x^{\mu}-k'_{\mu}y^{\mu})}\D^{3}y\,\D^{3}x\nonumber\\
&=-\I\int\left(\I\omega_{\vec{k}}\delta^{(3)}(\vec{x}-\vec{y})
+\I\omega_{\vec{k}'}\delta^{(3)}(\vec{y}-\vec{x})\right)\E^{\I (k_{\mu}x^{\mu}-k'_{\mu}y^{\mu})}\D^{3}y\,\D^{3}x
\end{align}
Great, performing the integral over $x$ gives us
\begin{equation}
\begin{split}
\int&\left(\omega_{\vec{k}}\delta^{(3)}(\vec{x}-\vec{y})
+\omega_{\vec{k}'}\delta^{(3)}(\vec{y}-\vec{x})\right)\E^{\I (k_{\mu}x^{\mu}-k'_{\mu}y^{\mu})}\D^{3}y\,\D^{3}x\\
&=\int(\omega_{\vec{k}}+\omega_{\vec{k}'})\E^{\I(k_{\mu}-k'_{\mu})y^{\mu}}\D^{3}y
\end{split}
\end{equation}
Again, this is just the Fourier transform of a simpler expression, since
the only place $y$ appears is in the exponential factor (whose integral
gives us a delta expression):
\begin{equation}
\int(\omega_{\vec{k}}+\omega_{\vec{k}'})\E^{\I(k_{\mu}-k'_{\mu})y^{\mu}}\D^{3}y
=2\omega_{\vec{k}}\cdot(2\pi)^{3}\delta^{(3)}(\vec{k}-\vec{k}').
\end{equation}
Hence we conclude
\begin{equation}
\commutator{a(k)}{a^{\dagger}(k')} = 2\omega_{\vec{k}}\cdot(2\pi)^{3}\delta^{(3)}(\vec{k}-\vec{k}').
\end{equation}
Observe the time component in the exponential factor
$\exp(\I(\omega_{\vec{k}}-\omega_{\vec{k}'})t)$ plays no role in the
final answer, because the delta function restricts this to when the
exponent vanishes (i.e., when $\omega_{\vec{k}}=\omega_{\vec{k}'}$).

(2) We see from Eq
\eqref{eq:proof:ladder-operator-commutation:expansion} that
\begin{equation}
\begin{split}
\commutator{a(k)}{a(k')}
=&\int\commutator{\omega_{\vec{k}}\varphi(\vec{x},t)}{\omega_{\vec{k}'}\varphi(\vec{y},t)}\E^{\I (k_{\mu}x^{\mu}-k'_{\mu}y^{\mu})}\D^{3}y\,\D^{3}x\\
&+\int\commutator{\omega_{\vec{k}}\varphi(\vec{x},t)}{\I\pi(\vec{y},t)}\E^{\I (k_{\mu}x^{\mu}-k'_{\mu}y^{\mu})}\D^{3}y\,\D^{3}x\\
&+\int\commutator{\I\pi(\vec{x},t)}{\omega_{\vec{k}'}\varphi(\vec{y},t)}\E^{\I (k_{\mu}x^{\mu}-k'_{\mu}y^{\mu})}\D^{3}y\,\D^{3}x\\
&+\int\commutator{\I\pi(\vec{x},t)}{\I\pi(\vec{y},t)}\E^{\I (k_{\mu}x^{\mu}-k'_{\mu}y^{\mu})}\D^{3}y\,\D^{3}x
\end{split}
\end{equation}
But only the second and third terms could survive, since momentum
commutes with itself (and the field commutes with itself, too). But we
see, by the exact same computations as before, we have a slight
difference in sign:
\begin{equation}\label{eq:proof:ladder-operator-commutation:vanishing-expansion}
\begin{split}
\int&\left(\commutator{\omega_{\vec{k}}\varphi(\vec{x},t)}{\I\pi(\vec{y},t)}-\commutator{\omega_{\vec{k}'}\varphi(\vec{y},t)}{\I\pi(\vec{x},t)}\right)\E^{\I (k_{\mu}x^{\mu}-k'_{\mu}y^{\mu})}\D^{3}y\,\D^{3}x\\
&=
\int(\omega_{\vec{k}}-\omega_{\vec{k}'})\E^{\I(k_{\mu}-k'_{\mu})y^{\mu}}\D^{3}y
\end{split}
\end{equation}
which vanishes.

(3) By the same token, we see
\begin{equation}
\begin{split}
\commutator{a^{\dagger}(k)}{a^{\dagger}(k')}
=&\int\commutator{\omega_{\vec{k}}\varphi(\vec{x},t)}{\omega_{\vec{k}'}\varphi(\vec{y},t)}\E^{\I (k_{\mu}x^{\mu}-k'_{\mu}y^{\mu})}\D^{3}y\,\D^{3}x\\
&-\int\commutator{\omega_{\vec{k}}\varphi(\vec{x},t)}{\I\pi(\vec{y},t)}\E^{\I (k_{\mu}x^{\mu}-k'_{\mu}y^{\mu})}\D^{3}y\,\D^{3}x\\
&-\int\commutator{\I\pi(\vec{x},t)}{\omega_{\vec{k}'}\varphi(\vec{y},t)}\E^{\I (k_{\mu}x^{\mu}-k'_{\mu}y^{\mu})}\D^{3}y\,\D^{3}x\\
&+\int\commutator{\I\pi(\vec{x},t)}{\I\pi(\vec{y},t)}\E^{\I (k_{\mu}x^{\mu}-k'_{\mu}y^{\mu})}\D^{3}y\,\D^{3}x
\end{split}
\end{equation}
which produces Eq
\eqref{eq:proof:ladder-operator-commutation:vanishing-expansion}
exactly, which we saw vanishes.
\end{proof}

\begin{xca}
Express the Hamiltonian in terms of $a(k)$, $a^{\dagger}(k)$. Recall
\begin{equation*}
H = \frac{1}{2}\int\left(\pi(x)^{2}+|\nabla\varphi(x)|^{2}+m^{2}\varphi(x)^{2}\right)\D^{3}x
\end{equation*}
\end{xca}
\begin{soln}
The real trick is to consider the commutation relations
\begin{equation}
\commutator{H}{\varphi(\vec{x}, t)}=\partial_{t}\varphi(\vec{x}, t)
\end{equation}
and how the commutator acts on the $a(k)$, $a^{\dagger}(k)$
operators. We then guess the form of $H$ based on the commutation
relations \Mref{thm:scalar:ladder-op-commutation-relations}.
\end{soln}

\N{Definition}
The \define{Ground State} (or ``\emph{Vacuum State}'' or just
``Vacuum'') is the guy in the Fock space $\ket{\vacuum}$ satisfying
\begin{equation}
a(k)\ket{0}=0,\quad\mbox{and}\quad \braket{\vacuum|\vacuum}=1.
\end{equation}
The state $a^{\dagger}(k)\ket{\vacuum}$ contains 1 particle of momentum
$\vec{k}$ and energy $\omega_{k}$. Similarly,
$a^{\dagger}(k)^{2}\ket{\vacuum}$ contains two such particles, and so
on.

\N{Physical States}
Lets try to take physical states as the ones which are normalizable,
i.e.,
\begin{equation}
0<|\braket{\mbox{phy}|\mbox{phy}}|<\infty.
\end{equation}
But look, $a^{\dagger}(k)\ket{\vacuum}$ is not normalizable:
\begin{subequations}
\begin{align}
\braket{\vacuum|a(k)a^{\dagger}(k)|\vacuum}
&= \braket{\vacuum|a^{\dagger}(k)a(k) + \commutator{a(k)}{a^{\dagger}(k)}|\vacuum}\\
&= \braket{\vacuum|a^{\dagger}(k)a(k)|\vacuum}
   + \braket{\vacuum|2\omega_{k}(2\pi)^{3}\delta(0)|\vacuum}\\
&= 0 + 2\omega_{k}(2\pi)^{3}\delta(0)
\end{align}
\end{subequations}
Hatfield~\cite{Hatfield:1992rz} argues this stray $\delta(0)$ is because
we are working with plane waves, which are not normalizable if the
volume containing the system is infinite. But I don't buy
this. Hatfield's reasoning appears to be based on the Fourier transform
of the delta function
\begin{equation}
\delta^{k}(x-x_{0}) = \frac{1}{(2\pi)^{k}}\int\E^{\I p\cdot(x-x_{0})}\D^{k}x
\end{equation}
then taking $x=x_{0}$ produces
\begin{equation}
\delta^{k}(0)\looksLike\int_{\RR^{k}}\D^{k}x
\end{equation}
which is precisely an integral over a given region.
I should rework everything in a finite box (with periodic boundary
conditions)\dots later. At any rate, to get normalizable states, we
should build a wave packet using superposition.

% 20

To be more precise, we smear\index{Smear}\index{Test Function!Smear} out
$a^{\dagger}(k)$. So the state
\begin{equation}
\ket{f} = \int f(\vec{k})a^{\dagger}(k)\ket{0}\,\D^{3}k
\end{equation}
will be normalizable, provided
\begin{equation}
0<\int|f(\vec{k})|^{2}\,\D^{3}k<\infty.
\end{equation}
So $a^{\dagger}(k)$ really only makes sense in an integral with a test
function $f(\vec{k})$. Hence $a^{\dagger}(k)$ is a distribution,
well---as alluded earlier---an ``operator-valued distribution''. So
doing an integral containing $a(k)$ gives us an operator (whereas an
integral containing $\delta(x)$ results in a number). Wearing our
physicist's cap, we will be sloppy and continue calling such things
``operators''.

We find the vacuum $\ket{\vacuum}$ is an eigenstate of the
Hamiltonian. What is its energy? We find:
\begin{equation}
\begin{split}
\braket{\vacuum|H|\vacuum}
&=\frac{1}{2}\int\braket{\vacuum|\omega_{k}(0+a(k)a^{\dagger}(k))|\vacuum}\frac{\D^{3}k}{(2\pi)^{3}}\\
&=\frac{1}{2}\int\omega_{k}\delta(0)\frac{\D^{3}k}{(2\pi)^{3}}
\end{split}
\end{equation}
As noted earlier, Hatfield equates $(2\pi)^{3}\delta^{3}(0)=\int\D^{3}x$
and asserts it is related to the infinite volume of the system, but in a
box\dots we'd get a similar divergence.

At any rate, the trick is to reorder the Hamiltonian and throw away the
divergent bit. We do this through normal ordering, which puts creation
operators to the left of the annihilation operators:
\begin{equation}
\normalOrder{a^{\dagger}(k)a(k)}
=\normalOrder{a(k)a^{\dagger}(k)}
=a^{\dagger}(k)a(k)
\end{equation}
Normal ordering is indicated by ``$\normalOrder{-}$''. The normally
ordered Hamiltonian is then
\begin{equation}
\normalOrder{H}=\int\omega_{k}a^{\dagger}(k)a(k)\frac{1}{2\omega_{k}}\frac{\D^{3}k}{(2\pi)^{3}}.
\end{equation}
Observe  then that
\begin{equation}
\braket{\vacuum|\normalOrder{H}|\vacuum}=0.
\end{equation}
Without loss of generality, we will assume that the Hamiltonian is
always normal ordered.

\N{Lorentz Invariance}
We started with a classical relativistic field, then seemingly broke
Lorentz invariance by using equal-time commutators. Nuts! We can
rephrase this as ``To quantize, we must choose a specific Lorentz
frame'', i.e., quantization is frame-dependent. We want Lorentz
invariance! One way to do this is to verify the quantum operator forms
of the generators of the Lorentz algebra still form a proper algebra
after quantization.

We should recall the basic structure of the Lorentz group. Confusingly,
most physicists refer to its Lie algebra as the ``Lorentz group''. We
will not. What symmetries does the Lorentz group consist of? There are
three:
\begin{enumerate}
\item Translations $x^{\mu}\to x^{\mu} + h^{\mu}$ for some constant
  $h^{\mu}$ displacement;
\item Spatial Rotations $x^{j}\to {R^{j}}_{k}x^{k}$ for $k=1,2,3$;
\item Lorentz Boosts $x^{t}\to {\Lambda^{t}}_{\mu}x^{\mu}$
\end{enumerate}
Physicists really care about the Lie algebra (``infinitesimal
symmetries'' which are ``close'' to the identity transformation).

We expect the field momentum operator (not to be confused with the
conjugate momenta) should be
\begin{equation*}
P^{\mu}\sim\sum_{k} k^{\mu}
  \begin{pmatrix}\mbox{number of}\\
    \mbox{particles in}\\
    \mbox{state $k^{\mu}$}
  \end{pmatrix}.
\end{equation*}
We can make this more rigorous by writing it out as
\begin{equation}
P^{\mu} = \int k^{\mu}a^{\dagger}(k)a(k)\frac{1}{2\omega_{k}}\frac{\D^{3}k}{(2\pi)^{3}}
\end{equation}
This is our definition for $P^{\mu}$.

\N*{Puzzle:} What is the field momentum operator in terms of $\varphi(x)$
and $\pi(x)$?

\N*{Solution:} After a bit of gymnastics, we find
\begin{equation}
  P^{j} = -\int\pi(x)\partial_{j}\phi(x)\,\D^{3}x.
\end{equation}

\N{Faster-Than-Light Problem?}
We \emph{really} don't want signals traveling faster than light. What to
do? We could specify commutators between fields vanish for space-like
intervals, i.e., for points $x$ and $y$ such that $(x-y)^{2}<0$. We
consider what this means, exactly, using the expansions:
\begin{subequations}
  \begin{align}
    \commutator{\varphi(x)}{\varphi(y)}
    &= \int\left(\commutator{a(k)}{a^{\dagger}(k')}\E^{-\I k\cdot
      x}\E^{-\I k'\cdot y} + \commutator{a^{\dagger}(k)}{a(k')}\E^{\I
      k\cdot x}\E^{-\I k'\cdot
      y}\right)\frac{1}{2\omega_{k}}\frac{1}{2\omega_{k'}}\frac{\D^{3}k'}{(2\pi)^{3}}\frac{\D^{3}k}{(2\pi)^{3}}\\
&=\int\frac{1}{2\omega_{k}}(\E^{-\I k\cdot(x-y)}-\E^{\I k\cdot(x-y)})\frac{\D^{3}k}{(2\pi)^{3}}\\
&\eqqcolon \I\Delta(x-y) = \I\Delta(x-y; m)
  \end{align}
\end{subequations}
where in that last line, we define $\Delta(x-y)$ and note that sometimes
we include the mass $m$ as a parameter.

\N*{Claim} The quantity $\Delta(x-y)$ is Lorentz-invariant.

\N*{Observation}
We see $\commutator{\varphi(x)}{\varphi(y)}=0$ when $x^{t}=y^{t}$,
implies $\Delta(x-y)=0$ when $x^{t}=y^{t}$.

\N*{Conclusion} We find $\commutator{\varphi(x)}{\varphi(y)}=0$ when
$(x-y)^{2}<0$ since it's true at equal times. Hence we can Lorentz boost
to see this describes arbitrary space-like intervals.

\N{Definition}
The \define{Pauli--Jordan Operator}\index{Pauli--Jordan Operator|textbf}
is precisely 
\begin{equation}
\Delta(x-y) = -\I\commutator{\varphi(x)}{\varphi(y)}.
\end{equation}
The \define{Pauli--Jordan Function}\index{Pauli--Jordan Function|textbf}
is $\braket{\vacuum|\Delta(x-y)|\vacuum}$.

\begin{remark}
For more on this, see Ryder~\cite[\S4.2 and \S6.1]{Ryder:1985wq}
\end{remark}

\N{Definition}
The property $\commutator{\varphi(x)}{\varphi(y)}=0$ for space-like
$(x-y)^{\mu}$ is called \define{Microscopic Causality}\index{Microscopic Causality}
(or ``\emph{Micro-Causality\/}''). 

(See Wright~\cite{wright2012thesis} for a full discussion of the
philosophical consequences of microcausality.)

\N{Remark}
The microcausal proeprty suggests we can measure the field at two 
space-like separated points exactly and ``simultaneously'' without 
interference or having one measurement disturbing the other. On 
the other hand, we cannot have terms like $\varphi(\vec{x})^{2}$ 
without causing horrible problems. For this reason, we 
smear\index{Smear}\index{Smearing Fields}\index{Fields!Smearing} 
field quantities, opting for 
$f(\vec{x}, \vec{y})\varphi(\vec{x})\varphi(\vec{y})$ 
instead.

\begin{theorem}[Interesting Properties of $\Delta$]
The function $\Delta(x-y)$ has the following properties:
\begin{enumerate}
\item It is odd: $\Delta(x-y)=-\Delta(y-x)$
\item $\left.\partial_{t}\Delta(x-y)\right|_{x^{t}=y^{t}} = -\delta^{3}(x-y)$.
\item It satisfies the Klein--Gordon equation: $(\Box + m^{2})\Delta(x)=0$.
\end{enumerate}
\end{theorem}

% pg 23
\M

\N{Remark}

\section{Charged Scalar Fields}

\N{Antiparticles}

\N{Canonical Formalism With Hermitian Fields}

\N{Canonical Formalism With Complex Fields}

\N{Plane Wave Expansion}

\N{$\U{1}$-Invariance}

\N{Remark}

\N{Remark}

\N{Remark}

\section{Time-Ordering and Propagators}

\N{Motivating Example}

\N{Definition}

\M

\N{Remark}
