\section{Second Quantization of Nonrelativistic Schrodinger Equation}
\M
There are two routes to ``second quantization'':
\begin{subequations}
\begin{equation}
\begin{pmatrix}
\mbox{Quantum}\\
\mbox{Mechanical}\\
\mbox{System}
\end{pmatrix}
\xrightarrow{\text{pretend it's a}}
\begin{pmatrix}
\mbox{Classical}\\
\mbox{Field}
\end{pmatrix}
\xrightarrow{\text{quantize!}}
\mbox{(QFT)}
\end{equation}
and
\begin{equation}
\begin{pmatrix}
\mbox{Classical}\\
\mbox{Field}
\end{pmatrix}
\xrightarrow{\text{quantize!}}
\mbox{(QFT)}.
\end{equation}
\end{subequations}
We really have two possible starting points along one path: are we given
a classical field (e.g., electromagnetism) or a quantum mechanical
system (e.g., a Hydrogen atom)? For the former, we interpret the wave
function $\psi(x)$ as a classical field.

But why would we ever really care about second quantization? If we have
a quantum system with a variable number of particles, for example, we
would need to use the second quantization formalism.

\N{Conventions}
Since this is quantum field theory, we will use the ``mostly minuses''
convention $(+---)$ so energy is positive. We will use Einstein
summation convention
\begin{equation}
a^{\mu}b_{\mu}=\sum_{\mu}a^{\mu}b_{\mu},
\end{equation}
i.e., we sum over this only when one index is ``upstairs'' and the other
``downstairs''. If we need to use Euclidean summation convention, where
sum over \emph{any} repeated index (both can be upstairs
$a^{\mu}b^{\mu}=\sum_{\mu}a^{\mu}b^{\mu}$, or downstairs, or mixed).

We suppose that spacetime is four-dimensional, unless working with toy
models in smaller dimensions. We reserve Greek letter indices for
spacetime indices.

\subsection{Schrodinger Equation}
Lets work with 1 spatial dimension, for simplicity. We will second
quantize the Schrodinger equation
\begin{equation}
\I\partial_{t}\varphi = \frac{-1}{2}\partial_{x}^{2}\varphi + V(x)\varphi
\end{equation}
where $0\leq x\leq L$, with periodic boundary conditions\footnote{I
  always wondered why physicists adopt periodic boundary conditions. The
  reason, I think, is because the circle $S^{1}$ is equivalent to the
  real line plus a point at infinity, provided that $-\infty$ is
  equivalent to $+\infty$.}. If we try to pretend the Schrodinger
equation may be derived from some action principle, we find it easier to
derive it from symmetry principles. 

\N{Conserved Currents in General}\label{subsec:schrodinger:symmetry}
Suppose we have $N$ fields, and we keep track of them with an index
$\varphi^{a}$ for $a=1$, \dots, $N$. We suppose the fields are invariant
under some symmetry transformation, which for ``small transformations''
generically\footnote{This is a bit ambiguous here. Typically the
  symmetry transformations form a representation for a Lie group, and ``small transformations'' mean ones close to the identity transformation --- so they take guys from the Lie algebra.} looks like
\begin{equation}
\varphi^{a}\to\varphi^{a}+\epsilon\varDelta^{a}(\varphi).
\end{equation}
where $0<\epsilon\ll1$ is a constant (an ``infinitesimal parameter'').
Now, we treat this as a variation in the field, and find the
corresponding variation in the action generated by this is
\begin{subequations}
\begin{align}
\delta \action[\varphi^{a}]
&=\int\sum_{a}\epsilon\left[\frac{\partial\mathcal{L}}{\partial(\partial_{\mu}\varphi^{a})}
\partial_{\mu}\varDelta^{a}(\varphi)
+\frac{\delta\mathcal{L}}{\partial\varphi^{a}}\varDelta^{a}(\varphi)\right]\D x\,\D t\\
\intertext{then plug in the Euler-Lagrange equations for the second-term}
\delta\action[\varphi^{a}]&=\int\sum_{a}\epsilon\partial_{\mu}\left[
\frac{\partial\mathcal{L}}{\partial(\partial_{\mu}\varphi^{a})}\varDelta^{a}(\varphi)\right]\D x\,\D t.
\end{align}
\end{subequations}
For this to vanish, which \emph{should} happen for physical and mathematical
reasons, we find we have a conserved current associated with the
symmetry $\varphi^{a}\to\varphi^{a}+\epsilon\varDelta^{a}(\varphi)$
(i.e., only of the fields among themselves and not some rotation of
spacetime coordinates)
\begin{equation}
j^{\mu}=\frac{\partial\mathcal{L}}{\partial(\partial_{\mu}\varphi)}.
\end{equation}
More generally, when we also treat $\mathcal{L}(x)$ as a scalar field in
its own right, then $x\to x+\delta x$ produces a transformation up to a
divergence
\begin{equation}
\mathcal{L}(x)\to\mathcal{L}(x)+\delta x^{\mu}\,\partial_{\mu}\mathcal{T}^{\mu}
\end{equation}
which means the most general conserved current would be
\begin{equation}
j^{\mu}=\frac{\partial\mathcal{L}}{\partial(\partial_{\mu}\varphi)}-\mathcal{T}^{\mu}.
\end{equation}
Great, so what? Well, lets try to derive an expression
\begin{equation}
\partial_{\mu}j^{\mu}=0
\end{equation}
from Schrodinger's equation.

\N{Schrodinger's Conserved Current}\label{par:schrodinger-conserved-current}
Lets multiply Schrodinger's equation by $\varphi^{*}$ (the complex
conjugate of the wave function $\varphi$):
\begin{equation}
\I\varphi^{*}(x,t)\partial_{t}\varphi(x,t)=\frac{-1}{2}\varphi^{*}(x,t)\partial_{x}^{2}\varphi(x,t)
+V(x)\varphi^{*}(x,t)\varphi(x,t).
\end{equation}
Now lets symmetrize it, in the sense of subtracting from the LHS its
complex conjugate, and likewise for the RHS. We see
\begin{subequations}
\begin{equation}
[\mbox{LHS}]^{*}=
-\I\varphi(x,t)\partial_{t}\varphi^{*}(x,t)
\end{equation}
and, assuming $V(x)$ is a real function,
\begin{equation}
[\mbox{RHS}]^{*}
=\frac{-1}{2}\varphi(x,t)\partial_{x}^{2}\varphi^{*}(x,t)
+V(x)\varphi(x,t)\varphi^{*}(x,t).
\end{equation}
\end{subequations}
Then we find
\begin{equation}
\begin{split}
\I\partial_{t}\bigl(\varphi(x,t)\varphi^{*}(x,t)\bigr)
&= \frac{-1}{2}\bigl(-\varphi^{*}(x,t)\partial_{x}^{2}\varphi(x,t)+\varphi(x,t)\partial_{x}^{2}\varphi^{*}(x,t)\bigr)\\
&=\frac{1}{2}\partial_{x}\bigl(\varphi^{*}(x,t)\partial_{x}\varphi(x,t)-
\varphi(x,t)\partial_{x}\varphi^{*}(x,t)\bigr).
\end{split}
\end{equation}
We now have our natural candidates
\begin{subequations}\label{eq:schrodinger:symmetry:candidates}
\begin{equation}
j^{0}=\varphi(x,t)\varphi^{*}(x,t)
\end{equation}
and
\begin{equation}
j^{1}=\frac{\I}{2}\bigl(\varphi^{*}(x,t)\partial_{x}\varphi(x,t)-
\varphi(x,t)\partial_{x}\varphi^{*}(x,t)\bigr)
\end{equation}
\end{subequations}

\N{Lagrangian}\label{subsec:schrodinger:lagrangian}
Now, taking the results of Eq \eqref{eq:schrodinger:symmetry:candidates},
we plug it into our equation for 
\begin{equation}
j^{\mu}=\frac{\partial\mathcal{L}}{\partial(\partial_{\mu}\varphi)}\varphi
\end{equation}
which immediately suggests that
\begin{equation}\label{eq:schrodinger:lagrangian}
\mathcal{L}=\frac{\I}{2}\bigl(\varphi^{*}\partial_{t}\varphi-\varphi\partial_{t}\varphi^{*}\bigr)
-\frac{1}{2}\partial_{x}\varphi^{*}\partial_{x}\varphi-V(x)\varphi^{*}\varphi.
\end{equation}
Observe $\delta\action/\delta\varphi^{*}=0$ gives us Schrodinger's
equation, and $\delta\action/\delta\varphi=0$ gives us the conjugate
equation.

\begin{xca}[Symmetry]
Show that the Lagrangian in Eq \eqref{eq:schrodinger:lagrangian} has the
global phase invariance symmetry $\varphi\to e^{\I\alpha}\varphi$
for any constant $\alpha$. 

Is the Lagrangian invariant under local phase invariance symmetry? I.e.,
if we let $\alpha=\alpha(x,t)$ be a smooth function of $x$ and $t$, is
the Lagrangian still invariant?
\end{xca}

\N{Canonical Formalism}\label{subsec:schrodinger:canonical}
We see the conjugate momenta to our system
\Mref{subsec:schrodinger:lagrangian} is given by
\begin{equation}\label{eq:schrodinger:momenta-density}
\pi(x,t)=\frac{\partial\mathcal{L}}{\partial(\partial_{t}\varphi)}=\I\varphi^{*}(x,t).
\end{equation}
So we have a nifty interpretation of $\pi\looksLike\varphi^{*}$. Then we
immediately have the quantum commutation relations
\begin{equation}\label{eq:schrodinger:ccr}
\begin{split}
\commutator{\varphi(x,t)}{\varphi^{*}(x',t)}=\delta(x-x')\\
\commutator{\varphi^{*}}{\varphi^{*}}=\commutator{\varphi}{\varphi}=0.
\end{split}
\end{equation}
The Hamiltonian is given by the Legendre transform of Eq
\eqref{eq:schrodinger:lagrangian}
\begin{subequations}\label{eq:schrodinger:hamiltonian}
\begin{align}
H&=\int\bigl(\pi(x,t)\partial_{t}\varphi(x,t)-\mathcal{L}\bigr)\,\D x\\
&=\int\left(\bigl[\I\varphi^{*}\partial_{t}\varphi\bigr]
-\bigl[\frac{\I}{2}\bigl(\varphi^{*}\partial_{t}\varphi-\varphi\partial_{t}\varphi^{*}\bigr)
-\frac{1}{2}\partial_{x}\varphi^{*}\partial_{x}\varphi-V(x)\varphi^{*}\varphi\bigr]\right)\D
x\label{eq:schrodinger:hamiltonian:substitution}\\
&=\int\left[\frac{1}{2}\partial_{x}\varphi^{*}\partial_{x}\varphi+V(x)\varphi^{*}\varphi\right]\D
x\label{eq:schrodinger:hamiltonian:integrate-by-parts}\\
&=\int\varphi^{*}\left[\frac{-1}{2}\partial_{x}^{2}+V(x)\right]\varphi\,\D
x.
\end{align}
\end{subequations}
Note: we get from \eqref{eq:schrodinger:hamiltonian:substitution}
to \eqref{eq:schrodinger:hamiltonian:integrate-by-parts} by rewriting
$\varphi^{*}\partial_{t}\varphi$ as
\begin{equation}
\varphi^{*}\partial_{t}\varphi =
\frac{\varphi^{*}\partial_{t}\varphi}{2}+\frac{\varphi^{*}\partial_{t}\varphi}{2}
\end{equation}
then integrating by parts on one of the terms on the right hand
side. This kills the time derivative components of the Lagrangian
density.

\N[Primary Constraints]{Remark}
We see that the mechanical system with Lagrangian $L=x\dot{y}-\dot{x}y$
must have primary constraints since its Hessian matrix $(\partial
p/\partial q)_{jk}=(0)_{jk}$ is the zero matrix. And this is the
mechanical counterpart to the system we are considering!

The Hessian matrix formed by the conjugate momenta density \eqref{eq:schrodinger:momenta-density}
is likewise the zero matrix. This tells us immediately we have primary
constraints. We should have
\begin{subequations}
\begin{equation}
C = \pi(x,t)-\I\varphi^{*}(x,t)=0
\end{equation}
be one primary constraint, and
\begin{equation}
C^{*} = \pi^{*}(x,t)+\I\varphi(x,t)=0
\end{equation}
\end{subequations}
be the other. Despite the notation, these are 
\emph{two independent constraints}.

\N{Remark}
Note in \Mref{subsec:schrodinger:canonical} we tacitly picked normal
operator ordering; if we did not, we'd have extra terms contributing
some ``vacuum energy''. Why exclude them? They're divergent terms
proportional to $\delta(0)$.

\N{Hamiltonian Dynamics}
So we get our Hamiltonian and it's remarkably simple. All that's left to
do is write $\varphi^{*}H\varphi$ and we're done, right? Err, well, no.

We need to treat $\varphi^{*}$ and $\varphi$ as independent
variables. When we impose the relations \eqref{eq:schrodinger:ccr} and
use the field Hamiltonian, we recover Schrodinger's equation when
imposing
\begin{equation}
\frac{\D}{\D t}\varphi = \I\commutator{H}{\varphi}
\end{equation}
Really? Well, we see that
\begin{subequations}
\begin{align}
\commutator{H}{\varphi}
&=\commutator{\int\varphi^{*}(x,t)\left(\frac{-1}{2}\partial_{x}^{2}+V(x)\right)\varphi(x,t)\,\D{x}}{\varphi(y,t)}\\
\intertext{lets introduce $h(x)$ satisfying
  $\mathcal{H}=\varphi^{*}h\varphi$, so}
\commutator{H}{\varphi}&=\int\commutator{\varphi^{*}(x,t)h(x)\varphi(x,t)}{\varphi(y,t)}\D x\\
&=\int\commutator{\varphi^{*}(x,t)}{\varphi(y,t)}h(x)\varphi(x,t)
   +\varphi^{*}(x,t)\commutator{h(x)\varphi(x,t)}{\varphi(y,t)}\D x\\
&=\int\delta(x-y)h(x)\varphi(x,t)\D x + \int\varphi^{*}(x,t)\cdot0\,\D{x}\\
&=h(y)\varphi(y,t) = \left(\frac{-1}{2}\partial_{y}^{2}+V(y)\right)\varphi(y,t)
\end{align}
\end{subequations}
and thus we recover Schrodinger's equation from the classical field
interpretation. 

\N{One-Body Hamiltonian} I'm still not convinced! Lets consider the first
quantized one-body Hamiltonian
\begin{equation}
\qop{h} = \frac{-1}{2}\partial_{x}^{2} + V(x)
\end{equation}
with normalized eigenfunctions $\varphi_{n}(x)$ and corresponding
eigenvalues $e_{n}$. We assume $\varphi_{n}$ form a complete
set\footnote{This is a trivial fact from linear algebra, supposing the
  $e_{n}$ are distinct. Then
  $\braket{\varphi_{m}|\qop{h}|\varphi_{n}}=e_{m}\braket{\varphi_{m}|\varphi_{n}}=e_{n}\braket{\varphi_{m}|\varphi_{n}}$
  which implies either $e_{m}=e_{n}$ or
  $\braket{\varphi_{m}|\varphi_{n}}=0$.}. So any solution of
Schrodinger's equation may be written out as
\begin{equation}
\varphi(x,t) = \sum_{n}a_{n}(t)\varphi_{n}(x).
\end{equation}
For the first quantized system, 
\begin{equation}
a_{n}=(\mbox{const})\exp(-\I e_{n}t),
\end{equation}
and $\varphi_{n}(x)$ and $\varphi(x,t)$ are wave functions. Second
quantization turns $\varphi(x,t)$ into an operator, so either $a_{n}(t)$
or $\varphi_{n}(x)$ must become an operator. (We make $a_{n}(t)$ the
operator. Similarly, we have $\varphi^{*}(x,t)=\sum
a^{\dagger}_{n}(t)\varphi^{*}_{n}(x)$.)

Lets see what happens when we substitute these eigenbasis expansions
into the commutation relations
\begin{subequations}
\begin{align}
\commutator{\varphi(x,t)}{\varphi^{*}(y,t)}
&=\commutator{\sum_{k}a_{k}(t)\varphi_{k}(x)}{\sum_{m}a^{\dagger}_{m}(t)\varphi^{*}_{m}(y)}\\
&=\sum_{k}\commutator{a_{k}(t)}{\sum_{m}a^{\dagger}_{m}(t)\varphi^{*}_{m}(y)}\varphi_{k}(x)\\
&=\sum_{k,m}\commutator{a_{k}(t)}{a^{\dagger}_{m}(t)}\varphi_{k}(x)\varphi^{*}_{m}(y)\\
&=\delta(x-y)
\end{align}
\end{subequations}
which happens iff
\begin{equation}
\commutator{a_{k}(t)}{a^{\dagger}_{m}(t)}=\delta_{km}.
\end{equation}
Similarly, the other commutation relations imply
\begin{equation}
\commutator{a_{k}(t)}{a_{m}(t)}=\commutator{a^{\dagger}_{k}(t)}{a^{\dagger}_{m}(t)}=0.
\end{equation}
Great.

So lets plug these eigen-expansions into the definition of the
Hamiltonian. We should note
\begin{equation}\label{eq:schrodinger:eigenbasis:completeness}
\int\varphi^{*}_{n}(x)\varphi_{m}(x)\,\D x=\delta_{mn}
\end{equation}
Now we have our series of manipulations
\begin{subequations}
\begin{align}
H
&=\int\left(\sum_{m}a^{\dagger}_{m}(t)\varphi^{*}_{m}(x)\right)
      \qop{h}(x)
      \left(\sum_{n}a_{n}(t)\varphi_{n}(x)\right)\D x\\
\intertext{then by Linearity}
H
&=\sum_{m,n}a^{\dagger}_{m}(t)a_{n}(t)\int\varphi^{*}_{m}(x)\qop{h}(x)\varphi_{n}(x)\,\D{x}\\
&=\sum_{m,n}a^{\dagger}_{m}(t)a_{n}(t)\int\varphi^{*}_{m}(x)\bigl(\qop{h}(x)\varphi_{n}(x)\bigr)\,\D{x}\\
&=\sum_{m,n}a^{\dagger}_{m}(t)a_{n}(t)\int\varphi^{*}_{m}(x)\bigl(e_{n}\varphi_{n}(x)\bigr)\,\D{x}\\
&=\sum_{m,n}a^{\dagger}_{m}(t)a_{n}(t)e_{n}\int\varphi^{*}_{m}(x)\varphi_{n}(x)\,\D{x}\\
\intertext{then using Eq \eqref{eq:schrodinger:eigenbasis:completeness}}
H
&=\sum_{m,n}a^{\dagger}_{m}(t)a_{n}(t)e_{n}\int\varphi^{*}_{m}(x)\varphi_{n}(x)\,\D{x}\nonumber\\
&=\sum_{m,n}a^{\dagger}_{m}(t)a_{n}(t)e_{n}\delta_{mn}\\
&=\sum_{n}a^{\dagger}_{n}(t)a_{n}(t)e_{n}
\end{align}
\end{subequations}
thanks to the Kronecker-delta. For a single fixed $n$, we
see\marginpar{\footnotesize\em$a^{\dagger}_{n} =$ creation operator;\\ $a_{n} =$
  annihilation operator}
$a^{\dagger}_{n}$ and $a_{n}$ are the raising and lowering
operators. Hence we see our field is just an infinite collection of
simple harmonic oscillators!

\N{Vacuum State}
The lowest energy state (that is still an eigenstate for $H$) is called
the \define{Vacuum State} (or ``\emph{Ground State}''), it is the one
which is ``empty'' in the sense that annihilation operators kill it
\begin{equation}\label{eq:schrodinger:def-of-vacuum}
a_{n}\ket{\vacuum} = 0.
\end{equation}
The destruction operator for any $n$ (i.e., $a_{n}$) finds no
excitations (``particles'') to annihilate in the vacuum, so the result
is the null vector. Similarly, $a^{\dagger}_{n}\ket{\vacuum}$ is a state of
energy $e_{n}$. It contains 1 particle of energy $e_{n}$, created by
$a^{\dagger}_{n}$, the creation operator for mode $n$. We have only one
field in this theory, so there is only one type of particle.

\begin{definition}
We can create a 2-particle state
$a^{\dagger}_{n}a^{\dagger}_{m}\ket{\vacuum}$ with energy $e_{m}+e_{n}$. The
collection of all states formed by creation operators acting on
$\ket{\vacuum}$ (for any $N$ --- and, technically, they can act on any
so-constructed state) produces a \define{Fock Space}.
\end{definition}

\N{Remarks}
(a) We emphasize here the abuse of notation going on in Eq
\eqref{eq:schrodinger:def-of-vacuum}. The right hand size \emph{is not a
  number}: it is a \emph{state}. Namely, it is the ``Null vector'' in
the Fock space.

(b) The Fock Space is defined as the \emph{closure} of all such states,
which doesn't mean much for physicists. But for mathematicians, this is
a very important property. Fock first introduced this type of space in
1932~\cite{fock1932}. 

(c) \textsc{Warning:} physicists often call Fock spaces ``infinite
dimensional Hilbert spaces''. This is wrong. A Hilbert space for, e.g.,
the free quantum particle \emph{is infinite dimensional}. What they mean
is a Fock space is the state space for an infinite number of particles.

\N{Particle Interpretation Puzzle} 
Since $a^{\dagger}_{n}(t)\ket{\vacuum}$ is a particle at time $t$ with
energy $e_{n}$, where is it located? We know $\varphi(x,t)$ is expanded
in terms of $a_{n}$ and $\varphi^{*}(x,t)$ is expanded in terms of
$a^{\dagger}_{n}(t)$. So $\varphi^{*}(x,t)\ket{\vacuum}$ gives a 1-particle
state located at $x$\dots but what's its energy? Or momentum?

\N{$V(x)=0$ Solution}
We see, when $V(x)=0$, the normalized eigenfunctions $\varphi_{n}(x)$
are plane waves
\begin{equation}
\varphi_{n}(x) = \frac{1}{\sqrt{L}}\exp(\I2\pi nx/L).
\end{equation}
That is to say, they are \emph{momentum eigenfunctions}. The expansions
of the wave functions
\begin{subequations}
\begin{align}
\varphi(x,t) &=\sum_{n}a_{n}(t)\varphi_{n}(x)\\
\intertext{and}
\varphi^{*}(x,t) &=\sum_{n}a^{\dagger}_{n}(t)\varphi^{*}_{n}(x)
\end{align}
\end{subequations}
are clearly Fourier series. We may deduce, then, that the coefficients are
\begin{subequations}
\begin{align}
a_{n}(t) &= \frac{1}{\sqrt{L}}\int\E^{-\I2\pi nx'/L}\varphi(x',t)\,\D
x'\\
a^{\dagger}(t)&= \frac{1}{\sqrt{L}}\int\E^{-\I2\pi nx'/L}\varphi^{*}(x',t)\,\D
x'
\end{align}
\end{subequations}
So what happens with the states? In this expansion, we see the state
\begin{equation}
a^{\dagger}_{n}(t)\ket{\vacuum} = \frac{1}{\sqrt{L}}\int
\E^{\I2\pi nx'/L}\varphi^{*}(x',t)\ket{\vacuum}\,\D x',
\end{equation}
which creates a particle at every point $x'$ with amplitude
$L^{-1/2}\exp(\I2\pi nx'/L)$. So we have the particle spread out over
space.

In fact, the probability is independent of $x'$, so it is equi-probable
of being anywhere. Is this sane? The particle has definite momentum $n$,
so by the uncertainty principle, we should have no clue what its position
is. By the same token, the state $\varphi^{*}(x,t)\ket{\vacuum}$ produces a
1-particle state at position $x$, but we do not know what its momentum is.

\N{$V(x)\neq0$ Solution}
Now, the more general case, when $V(x)\neq0$. Suppose we have a state
using a wave function $f(x,t)$. Then to create a particle at $x$, we sum
over the position
\begin{equation*}
\int f(x,t)\varphi^{*}(x,t)\ket{\vacuum}\,\D x.
\end{equation*}
Great, now what?

We demand this has definite energy $E$. So what? Well, the expression is
an eigenfucntion of the Hamiltonian:
\begin{equation}
H\int f(x,t)\varphi^{*}(x,t)\ket{\vacuum}\,\D x
=E\int f(x,t)\varphi^{*}(x,t)\ket{\vacuum}\,\D x.
\end{equation}
Observe:
\begin{subequations}
\begin{align}
H\int f(x,t)\varphi^{*}(x,t)\ket{\vacuum}\,\D x
&=\int\varphi^{*}(y,t)h(y)\varphi(y,t)\int f(x,t)\varphi^{*}(x,t)\ket{\vacuum}\,\D x\,\D{y}\\
&=\iint\varphi^{*}(y,t)h(y)\varphi(y,t)f(x,t)\varphi^{*}(x,t)\ket{\vacuum}\,\D x\,\D{y}\\
&=\iint\varphi^{*}(y,t)h(y)f(x,t)\Bigl(\varphi(y,t)\varphi^{*}(x,t)\Bigr)\ket{\vacuum}\,\D x\,\D{y}
\end{align}
then we commute the parenthesized terms
\begin{equation}
\begin{split}
\iint&\varphi^{*}(y,t)h(y)f(x,t)\Bigl(\varphi(y,t)\varphi^{*}(x,t)\Bigr)\ket{\vacuum}\,\D x\,\D{y}\\
&=\iint\varphi^{*}(y,t)h(y)f(x,t)\Bigl(\varphi^{*}(x,t)\varphi(y,t)+\commutator{\varphi(y,t)}{\varphi^{*}(x,t)}\Bigr)\ket{\vacuum}\,\D x\,\D{y}
\end{split}
\end{equation}
realizing the annihilation operator is acting on the vacuum
  in the first term in the parentheses simplifies things to
\begin{align}
\iint&\varphi^{*}(y,t)h(y)f(x,t)\Bigl(\varphi^{*}(x,t)\varphi(y,t)+\commutator{\varphi(y,t)}{\varphi^{*}(x,t)}\Bigr)\ket{\vacuum}\,\D x\,\D{y}\nonumber\\
&=\iint\varphi^{*}(y,t)h(y)f(x,t)\Bigl(0+\delta(x-y)\Bigr)\ket{\vacuum}\,\D x\,\D{y}\\
&=\int\varphi^{*}(x,t)h(x)f(x,t)\ket{\vacuum}\,\D x.
\end{align}
\end{subequations}
It follows we need
\begin{equation}
h(x)f(x,t) = \left(\frac{-1}{2}\partial_{x}^{2}+V(x)\right)f(x,t) = Ef(x,t),
\end{equation}
and hence
\begin{equation}
f(x,t) = \varphi_{n}(x)\exp(-\I e_{n}t).
\end{equation}
That is, we have recovered the quantized 1-particle system from quantum
field theory.

\N{Two-Body Schrodinger Equation}
For the 2-particle state, we see it resembles
\begin{equation}
\begin{pmatrix}
\mbox{state}
\end{pmatrix}
\looksLike
\varphi^{*}(x_{1},t)\varphi^{*}(x_{2},t)\ket{\vacuum}.
\end{equation}
If the two particles are spread out with wave function $f(x_{1}, x_{2},
t)$, then the state is
\begin{equation}
\begin{pmatrix}
\mbox{wave}\\\mbox{function}
\end{pmatrix}
\looksLike
\int f(x_{1}, x_{2}, t)\varphi^{*}(x_{1},t)\varphi^{*}(x_{2},t)\ket{\vacuum}\,\D{x_{1}}\D{x_{2}}
\end{equation}
We again apply the Hamiltonian to it and find
\begin{subequations}
\begin{align}
H&\int f(x_{1}, x_{2}, t)\varphi^{*}(x_{1},t)\varphi^{*}(x_{2},t)\ket{\vacuum}\,\D{x_{1}}\D{x_{2}}\nonumber\\
&=\iint\varphi^{*}(y,t)h(y)\varphi(y,t)f(x_{1}, x_{2}, t)\varphi^{*}(x_{1},t)\varphi^{*}(x_{2},t)\ket{\vacuum}\,\D{x_{1}}\D{x_{2}}\D{y}\\
&=\iint\varphi^{*}(y,t)h(y)f(x_{1}, x_{2}, t)\Bigl(\varphi(y,t)\varphi^{*}(x_{1},t)\varphi^{*}(x_{2},t)\Bigr)\ket{\vacuum}\,\D{x_{1}}\D{x_{2}}\D{y}
\end{align}
\end{subequations}
again commuting the parenthetic terms gives us
\begin{subequations}
\begin{align}
\Bigl(\varphi(y,t)\varphi^{*}(x_{1},t)\varphi^{*}(x_{2},t)\Bigr)
&=\Bigl(\varphi^{*}(x_{1},t)\varphi(y,t)+\commutator{\varphi(y,t)}{\varphi^{*}(x_{1},t)}\Bigr)\varphi^{*}(x_{2},t)\\
&=\Bigl(\varphi^{*}(x_{1},t)\varphi(y,t)+\delta(y-x_{1})\Bigr)\varphi^{*}(x_{2},t)\\
&=\varphi^{*}(x_{1},t)\varphi(y,t)\varphi^{*}(x_{2},t)+\delta(y-x_{1})\varphi^{*}(x_{2},t)\\
&=\varphi^{*}(x_{1},t)\Bigl(\varphi(y,t)\varphi^{*}(x_{2},t)\Bigr)+\delta(y-x_{1})\varphi^{*}(x_{2},t)\\
&=\varphi^{*}(x_{1},t)\Bigl(\varphi^{*}(x_{2},t)\varphi(y,t)+\commutator{\varphi(y,t)}{\varphi^{*}(x_{2},t)}\Bigr)\nonumber\\
&\phantom{=}\quad+\delta(y-x_{1})\varphi^{*}(x_{2},t)\\
&=\varphi^{*}(x_{1},t)\Bigl(\varphi^{*}(x_{2},t)\varphi(y,t)+\delta(y-x_{2})\Bigr)\nonumber\\
&\phantom{=}\quad+\delta(y-x_{1})\varphi^{*}(x_{2},t)
\end{align}
But only the delta terms survive (because $\varphi$ acts on the vacuum
state via the annihilation operators, which vanish), giving us
\begin{equation}
\Bigl(\varphi(y,t)\varphi^{*}(x_{1},t)\varphi^{*}(x_{2},t)\Bigr)
=\varphi^{*}(x_{1},t)\delta(y-x_{2})+\delta(y-x_{1})\varphi^{*}(x_{2},t).
\end{equation}
\end{subequations}
Then plugging this back into our monstrous integral gives us
\begin{subequations}
\begin{align}
&\iint\varphi^{*}(y,t)h(y)f(x_{1}, x_{2}, t)\Bigl(\varphi(y,t)\varphi^{*}(x_{1},t)\varphi^{*}(x_{2},t)\Bigr)\ket{\vacuum}\,\D{x_{1}}\D{x_{2}}\D{y}\\
&=\iint\varphi^{*}(y,t)h(y)f(x_{1}, x_{2}, t)
\Bigl(\varphi^{*}(x_{1},t)\delta(y-x_{2})+\delta(y-x_{1})\varphi^{*}(x_{2},t)\Bigr)
\ket{\vacuum}\,\D{x_{1}}\D{x_{2}}\D{y}\nonumber\\
&=\iint\varphi^{*}(x_{2},t)h(x_{2})\varphi^{*}(x_{1},t)f(x_{1},x_{2},t)\ket{\vacuum}\,\D{x_{1}}\D{x_{2}}\\
&\quad+
\iint\varphi^{*}(x_{1},t)h(x_{1})\varphi^{*}(x_{2},t)f(x_{1}, x_{2}, t)\ket{\vacuum}\,\D{x_{1}}\D{x_{2}}\nonumber\\
&=\iint\Bigl(h(x_{1})+h(x_{2})\Bigr)\varphi^{*}(x_{1},t)\varphi^{*}(x_{2},t)f(x_{1},x_{2},t)\ket{\vacuum}\,\D{x_{1}}\D{x_{2}}.
\end{align}
\end{subequations}
So it is an eigenstate of the Hamiltonian if
\begin{equation}
\left(\frac{-1}{2}\frac{\partial^{2}}{\partial x_{1}^{2}}
-\frac{1}{2}\frac{\partial^{2}}{\partial x_{2}^{2}}
+V(x_{1})+V(x_{2})\right)f(x_{1},x_{2},t) = Ef(x_{1},x_{2},t).
\end{equation}
Observe the 2-body potential separates and the two particles do not
interact. Hence we may perform a separation of variables
\begin{equation}
f(x_{1},x_{2},t)=g_{1}(x_{1},t)g_{2}(x_{2},t).
\end{equation}
Why is there no interaction? Because the Hamiltonian \eqref{eq:schrodinger:hamiltonian}
is quadratic in the field variables. We need to add a quartic term
\begin{equation}
\begin{pmatrix}
\mbox{Nontrivial}\\\mbox{Interaction}\\\mbox{Potential}
\end{pmatrix}
\looksLike\int V_{\text{int}}(x)\varphi^{*}(x,t)\varphi^{*}(x,t)\varphi(x,t)\varphi(x,t)\D{x}
\end{equation}
we see we get a 2-body potential interaction term
$2V_{\text{int}}(x_{1})\delta(x_{1}-x_{2})$, which we see the $\delta$ comes from
the commutation relations.

\N{Remarks} (a) The interaction is nonzero only if we have a contact
interaction (i.e., $x_{1}=x_{2}$).

(b) If we instead add $(\varphi^{*})^{3}\varphi^{3}$, then the 2-body
potential vanishes. We get instead a \emph{three-body} potential term
proportional to $\delta(x_{1}-x_{2})\delta(x_{2}-x_{3})$. Again,
interactions only occur when the three bodies meet.

\begin{definition}
When the first quantized $n$-body potential is a contact
interaction proportional to a $\delta$-function, then we say the quantum
field is \define{Local}. A quantum theory without this property is
called \define{Nonlocal}.

(Nonlocal fields are hard to construct and understand, especially when
they're relativistic!)
\end{definition}

% 2.10 on pg 10
\N{$N$-Body Schrodinger Equation}
If we apply the Hamiltonian to an $n$-particle state,
\begin{equation}
\int f_{n}(x_{1},\dots,x_{n})\varphi^{*}(x_{n},t)(\cdots)\varphi^{*}(x_{1},t)\ket{\vacuum}\,\D{x}_{1}
\cdots\D{x}_{n}
\end{equation}
we find the $n$-body Schrodinger equation
\begin{equation}\label{eq:schrodinger:n-body-hamiltonian}
\left(\sum^{n}_{j=1}\qop{h}(x_{j})\right)f_{n}(x_{1},\dots,x_{n})=E_{n}f_{n}(x_{1},\dots,x_{n}).
\end{equation}
In this way, all $n$-body Schrodinger equations are contained in the
second quantized system (i.e., quantum field theory).

One way to exactly solve a quantum field theory is to solve Eq
\eqref{eq:schrodinger:n-body-hamiltonian} for arbitrary $n$. Then any
amplitude corresponds to an integral over the initial and final wave
state functions. 


\N{Bosons}
We see $\varphi^{*}$ commutes with itself. Further, with the 2-particle
wave function 
\begin{equation*}
\int f_{2}(x_{1}, x_{2}, t)\varphi^{*}(x_{1},t)\varphi^{*}(x_{2},t)\ket{\vacuum}\,\D{x_{1}}\D{x_{2}}
\end{equation*}
we may commute $\varphi^{*}(x_{1},t)$ with $\varphi^{*}(x_{2},t)$ and
create identical particles. We interchange $x_{1}$ and $x_{2}$, yet we
get an \emph{identical} state. That is,
\begin{equation}
f_{2}(x_{1},x_{2},t) = f_{2}(x_{2},x_{1},t).
\end{equation}
By the Pauli Exclusion principle, this exchange symmetry implies we have
bosons. 

\N{Fermions}
What do we do if we wanted fermions instead? No two fermions can be at
exactly the same point (because no two fermions can have identical
states), so if $\varphi^{*}$ creates a fermion, then
\begin{equation}
\varphi^{*}(x,t)\varphi^{*}(x,t)\ket{\vacuum}=0.
\end{equation}
Since $\varphi^{*}(x,t)\ket{\vacuum}$ is not the null vector, then
\begin{equation}
\bigl(\varphi^{*}(x,t)\bigr)^{2}=0
\end{equation}
as an operator equation. If we use commutators, we must deduce that
\begin{equation}
\varphi^{*}(x,t)=0
\end{equation}
Something is clearly wrong. We ought to have
\begin{equation}
\bigl(\varphi^{*}\bigr)^{2}=0\quad\mbox{but}\quad\varphi^{*}(x,t)\neq0.
\end{equation}
How?

The wave function for fermions must be antisymmetric under the exchange
of coordinates
\begin{equation}
\psi_{2}(x_{1}, x_{2}, t)=-\psi_{2}(x_{2},x_{1},t).
\end{equation}
Consider a 2-fermion state
\begin{subequations}
\begin{align}
\ket{2f}
&= \int\psi_{2}(x_{1},x_{2},t)\varphi^{*}(x_{2},t)\varphi^{*}(x_{1},t)\ket{\vacuum}
\,\D x_{1}\D x_{2}\\
\intertext{then by antisymmetry}
\ket{2f}&= -\int\psi_{2}(x_{2},x_{1},t)\varphi^{*}(x_{2},t)\varphi^{*}(x_{1},t)\ket{\vacuum}
\,\D x_{1}\D x_{2}\\
\intertext{and relabeling the variables}
\ket{2f}&=-\int\psi_{2}(x_{1},x_{2},t)\varphi^{*}(x_{1},t)\varphi^{*}(x_{2},t)\ket{\vacuum}
\,\D x_{1}\D x_{2}.
\end{align}
\end{subequations}
For this to be nontrivial (i.e., not the null vector), we need
\begin{equation}
\varphi^{*}(x_{2},t)\varphi^{*}(x_{1},t)
=-\varphi^{*}(x_{1},t)\varphi^{*}(x_{2},t)
\end{equation}
We need $\varphi^{*}$ to \emph{anticommute} with itself.
If we write
\begin{equation}
\anticommute{A}{B} = AB+BA
\end{equation}
for the anticommutator, then the quantum conditions for the fermions
must be
\begin{subequations}
\begin{equation}
\anticommute{\varphi(x,t)}{\varphi^{*}(x',t)}=\delta(x-x')
\end{equation}
\begin{equation}
\anticommute{\varphi(x,t)}{\varphi(x',t)}=
\anticommute{\varphi^{*}(x,t)}{\varphi^{*}(x',t)}=0
\end{equation}
\end{subequations}

\subsection{Nonlinear Schrodinger Equation}
\M
The nonlinear Schrodinger equation is a $(1+1)$-dimensional quantum
field theory describing a nonrelativistic Bose gas. (A Bose gas is the
quantum analog of the classical ideal gas, but with Bosonic matter.) It
is based on the second quantization of the cubic Schrodinger equation
\begin{equation}
\I\partial_{t}\varphi = -\partial_{x}^{2}\varphi + 2c|\varphi|^{2}\varphi
\end{equation}
where $c>0$, we take $0\leq x\leq L$ and impose periodic boundary
conditions (we pretend the gas is confined to a box of length $L$). The
Hamiltonian is 
\begin{equation}
H = \int^{L}_{0}\left(\varphi^{*}(x,t)\bigl(-\partial_{x}^{2}\bigr)\varphi(x,t)+2c\varphi^{*}(x,t)\varphi^{*}(x,t)\varphi(x,t)\varphi(x,t)\right)\D{x}.
\end{equation}
The commutator relations recover the cubic Schrodinger equation with
periodic boundary conditions.

We recover the equivalent first quantized system by requiring the
$N$-body state
\begin{equation*}
\int f_{N}(x_{1},\dots,x_{N},t)\varphi^{*}(x_{N},t)(\cdots)\varphi^{*}(x_{1},t)\ket{\vacuum}\,\D{x_{1}}(\cdots)\D{x_{N}}
\end{equation*}
to be an eigenstate of the Hamiltonian. This means
\begin{equation}
\left(\sum^{N}_{j=1}-\partial_{j}^{2}+c\sum_{j\neq k}\delta(x_{j}-x_{k})\right)f_{N}
= E_{N}f_{N}.
\end{equation}
Periodic boundary conditions means
\begin{equation}
 f_{N}(x_{1},\dots,x_{k}=0,\dots,x_{N})
=f_{N}(x_{1},\dots,x_{k}=L,\dots,x_{N})
\end{equation}
for any $k=1,\dots,N$.

\begin{definition}
A quantum field theory possessing an infinite number of
conserved quantities is called \define{Integrable}
\end{definition}

\begin{theorem}
Integrable theories should be exactly solvable.
\end{theorem}

\N{$N$-Body System}
The general $N$-body system has a solution, first worked out by Lieb and
Liniger~\cite{Lieb:1963rt,Lieb:1963zz}. The general solution is
\begin{equation}
f_{N}(x_{1},\dots,x_{N})
=
\left(\prod_{j<\ell}
\theta(x_{j}-x_{\ell})+\E^{\I\Delta(k_{\ell}-k_{j})}\theta(x_{\ell}-x_{j})\right)
\exp(\I\sum^{N}_{j=1}k_{j}x_{j})
\end{equation}
where $\theta$ is the Heaviside step function, and
\begin{equation}
\E^{\I\Delta(q)}=\frac{q+\I c}{q-\I c}.
\end{equation}
For $f_{N}$ to be periodic, the $k$'s must satisfy
\begin{equation}
\E^{\I k_{j}L} = \prod_{\ell\neq j}\E^{\I\Delta(k_{\ell}-k_{j})}.
\end{equation}
The energy eigenvalue for this state is 
\begin{equation}
E_{N}=\sum^{N}_{j=1}k_{j}^{2}.
\end{equation}


\N{Thermodynamic Limit}
Lets consider the ``thermodynamic'' limit of the finite density
sector. Here we require the states to satisfy the constraint
\begin{equation}
\braket{f_{N}|\varphi^{*}(x)\varphi(x)|f_{N}}=\frac{N}{L}.
\end{equation}
The thermodynamic limit is specifically when we take $N\to\infty$
\emph{and} $L\to\infty$ \emph{but} we keep $N/L$ fixed as a finite real
number. 


\N{Operator Product Expansion}
Lets examine the 2-point time-independent field-field correlation
function
\begin{equation}
g(x-y)=\braket{\vacuum|\varphi^{*}(x)\varphi(y)|\vacuum}
\end{equation}
in the finite-density sector (i.e., where $g(0)=N/L$) to illustrate the
operator product expansion.

We Taylor expand $g(x-y)$ about $x=y$:
\begin{subequations}
\begin{align}
g(x-y)
&= g(0)
   + \partial_{x}g(x-y)|_{x=y}(x-y) 
   + \left.\frac{1}{2}\partial_{x}^{2}g(x-y)\right|_{x=y}(x-y)^{2}
   + \cdots\\
&= \braket{\vacuum|\varphi^{*}(y)\varphi(y)|\vacuum}\\
&\quad + \left.\braket{\vacuum|\partial_{x}\varphi^{*}(x)\varphi(y)|\vacuum}\right|_{x=y} (x-y)\nonumber\\
&\quad + \left.\frac{1}{2!}\braket{\vacuum|\partial_{x}^{2}\varphi^{*}(x)\varphi(y)|\vacuum}\right|_{x=y}
   (x-y)^{2}
   + \dots\nonumber
\end{align}
\end{subequations}
So lets try to examine these terms.

We see the first term is the number operator. 

The second term is a conserved quantity
\begin{equation}
M_{1} = \I\int\bigl(\partial_{x}\varphi^{*}(x)\bigr)\varphi(x)\,\D x
\end{equation}
which gives us
\begin{equation}
\left.\braket{\vacuum|\partial_{x}\varphi^{*}(x)\varphi(y)|\vacuum}\right|_{x=y}
=-\I\frac{M_{1}}{L}.
\end{equation}

We see the third term may be derived from the Hamiltonian
\begin{equation}
M_{2} = H = \int\left(\bigl(-\partial^{2}\varphi^{*}(x)\bigr)\varphi(x)
+ c \bigl(\varphi^{*}(x)\varphi(x)\bigr)^{2}\right)\D x
\end{equation}
so
\begin{equation}
\left.\braket{\vacuum|\partial_{x}^{2}\varphi^{*}(x)\varphi(y)|\vacuum}\right|_{x=y}
=\frac{-E_{0}}{L}+\frac{c}{L}\braket{\vacuum|\varphi^{*}(x)\varphi^{*}(x)\varphi(x)\varphi(x)|\vacuum}.
\end{equation}

Combining terms together, we find
\begin{equation}
\begin{split}
g(x-y)&=\frac{N}{L}-\I\frac{M}{L}(x-y)\\
&\quad+\frac{1}{2}\left(\frac{-E_{0}}{L}+\frac{c}{L}\braket{\vacuum|\varphi^{*}(x)\varphi^{*}(x)\varphi(x)\varphi(x)|\vacuum}\right)(x-y)^{2}.
\end{split}
\end{equation}
We see at higher order terms, the coefficient consists of conserved
quantities and higher point functions evaluated at the same point (i.e.,
we avoid all unpleasant nonlocality problems).

The operator production expansion then becomes
\begin{equation}
\begin{split}
\varphi^{*}(x)\varphi(y)\xrightarrow{x\to y}&
\left(1 - \I\frac{M_{1}}{N}(x-y)-\frac{1}{2}\frac{E_{0}}{N}(x-y)^{2}+\cdots\right)\varphi^{*}(y)\varphi(y)\\
&+\frac{1}{2}\frac{c}{L}(x-y)^{2}\varphi^{*}(y)\varphi^{*}(y)\varphi(y)\varphi(y)+\cdots
\end{split}
\end{equation}
This series terminates at the $(N+1)^{\text st}$ term for the finite
$N$-body system. The computation of $g(x-y)$ in closed form in the
thermodynamic limit remains an open problem.

\N{References}
One should also refer to Fetter and Walecka~\cite[Ch.\ 1]{fetter} 
for more examples of many-particle systems. Furthermore, trying to 
connect the spin-statistics theorem to non-relativistic quantum 
theory seems quite wrong (see Allen and Mondragon~\cite{allen2003ss}).
