\makeatletter
% notation
\@ifclassloaded{svmono}{}{%
  \newcommand{\E}{\mathrm{e}}
  \let\eul=\E
  \newcommand{\I}{\mathrm{i}}
  \let\imag=\I
  \newcommand{\D}{\mathrm{d}}
  \def\tens#1{\relax\ifmmode\mathsf{#1}\else\textsf{#1}\fi}
  \def\vec#1{\ensuremath{\mathchoice
                       {\mbox{\boldmath$\displaystyle#1$}}
                       {\mbox{\boldmath$\textstyle#1$}}
                       {\mbox{\boldmath$\scriptstyle#1$}}
                       {\mbox{\boldmath$\scriptscriptstyle#1$}}}}}
\DeclareMathOperator{\sgn}{sgn}

% calculus

% linear algebra
\DeclareMathOperator{\rank}{rank}

% linear algebra
\DeclareMathOperator{\rank}{rank}

% various equalities
\newcommand\looksLike{\sim}
\newcommand\orderOf{\sim}
\newcommand\bigO[1]{\ensuremath{\mathcal{O}\left(#1\right)}}
\newcommand\eqDef{\coloneqq}

% qft specific notation
\newcommand\vacuum{0}
\@ifundefined{DeclarePairedDelimiter}{%
  \newcommand\normalOrder[1]{\mathrel{:}\!#1\!\mathrel{:}}%
}{%
  \DeclarePairedDelimiter{\normalOrder}{\vcentcolon}{\vcentcolon}%
}

% common sets
\newcommand\CC{\mathbb{C}}
\newcommand\NN{\mathbb{N}}
\newcommand\QQ{\mathbb{Q}}
\newcommand\RR{\mathbb{R}}
\newcommand\ZZ{\mathbb{Z}}

% common groups
\newcommand\SU[1]{\operatorname{SU}(#1)}
\newcommand\U[1]{\operatorname{U}(#1)}
\newcommand\classicalGroup[1]{\mathsf[1]}

% commutators & brackets
\newcommand\anticommute[2]{\ensuremath{\left\{#1,\;#2\right\}}}
\newcommand\comm[2]{\ensuremath{\left\lbrack#1,\;#2\right\rbrack}}
\let\anticommutator\anticommute
\let\commutator\comm
\newcommand\PoissonBracket[2]{\ensuremath{\left\{#1,\;#2\right\}}}
\newcommand\DiracBracket[2]{\PoissonBracket{#1}{#2}_{D}}

% notation
\newcommand\action{\ensuremath{S}}
\newcommand\qop[1]{\widehat{#1}}
\newcommand\stepFn{\mathop{\theta}\nolimits}
\newcommand\metric{\ensuremath{g}}
\newcommand\weakEq{\approx}
\newcommand\extendedHamiltonian{\ensuremath{H_{E}}}
\newcommand\totalHamiltonian{\ensuremath{H_{T}}}
\newcommand\firstClassHamiltonian{\ensuremath{H'}}
\newcommand\firstClassConstraint{\gamma}
\newcommand\secondClassConstraint{\chi}
\makeatother
